\chapter*{Abstract}

\emph{Financial time series are known for their non-stationary behaviour and
this has motivated its study using different techniques. 
One relevant observation is that sometimes time series exhibit some stationary
linear combinations. When this happens, it is said that those time series are
cointegrated.  Cointegration has been then one of the main features studied.
Vector error correction model (VECM) is an econometric model which characterizes
the joint dynamic behaviour of a set of cointegrated variables in terms of
forces pulling towards equilibrium.}

\emph{VECM parameters are obtained using the ordinary least squares (OLS)
method.  Even though OLS is extensively used, it has over-fitting
issues and is computationally expensive. Ridge regression is commonly used
instead of OLS since they include a regularization parameter which could improve
prediction error.}

\emph{In this thesis, financial time series features are studied and different
algorithms were developed in order to optimize parameters an increase performance.
In particular, we propose an online VECM which optimizes how model parameters
are obtained considering only past data and different ways of solving the model
using machine learning techniques.  
The long-run relationship between the time series is used in order to make
optimizations and obtain improved execution times. However, due to the large
amount of data available and the need of quick response, algorithms presented
were optimized using high performance computing.
Our experiments were tested using synthetic data and foreign exchange rates.
Results show that cointegration and high performance computing allow to obtain
better models and execution times.}

\vspace{0.8cm}

\emph{\textbf{Keywords:} Computational finance  - Cointegration  - Time Series - 
Online algorithms - Resampling  - Regression}


