\chapter*{Abstract}

\emph{Financial time series are known for their non-stationary behaviour and
this has motivated its study using different techniques. 
One relevant observation is that sometimes time series exhibit some stationary
linear combinations. When this happens, it is said that those time series are
cointegrated.  Cointegration has been then one of the main features studied.
Vector error correction model (VECM) is an econometric model which characterises
the joint dynamic behaviour of a set of cointegrated variables in terms of
forces pulling towards equilibrium.}

\emph{VECM parameters are obtained using the ordinary least squares (OLS)
method.  Even though OLS is extensively used, it has over-fitting
issues and is computationally expensive. Ridge regression is commonly used
instead of OLS since they include a regularisation parameter which could improve
prediction error.}

\emph{In this thesis, financial time series features are studied and different
algorithms were developed in order to optimise parameters and increase performance.
In particular, I propose an online algorithm based on VECM which optimises how model parameters are obtained and reduces execution times. This is achieved considering only a sliding window of the last historical data and using machine learning techniques to solve the model. Moreover, the long-run relationship between the time series is used in order to make
optimisations and obtain improved execution times. 
Due to the large amount of financial data available and the need of quick response, 
the algorithm presented
was optimised using high performance computing.
Our experiments were tested using synthetic data and foreign exchange rates.
Results show that cointegration and high performance computing allow to obtain
models with better performance accuracy and reduced execution times.}

\vspace{0.8cm}

\emph{\textbf{Keywords:} Computational finance  - Cointegration  - Time Series - 
Online algorithms - Resampling  - Regression}


