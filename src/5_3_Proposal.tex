\chapter{Real-time intraday Forex rates forecasting using a regularized Vector
Error Correction model}

Vector error correction model (VECM) parameters are commonly obtained using the
ordinary least squares method. Since this parameters could be sensitive to the
data, this proposal is to get these parameters using ridge regression which
could lead to a better generalization capability. In order to give with a
real-time response, an online version of VECM (OVECM) is proposed. OVECM solves
VECM considering only a sliding window of historical data and therefore
better response times could be achieved.


\section{The problem}

VECM introduces the long-run relationship among a set of cointegrated variables
as an error correction term. VECM is a special case of the vector autorregresive
model (VAR) model. VAR model expresses future values as a linear combination of
variables past values.  However, VAR model cannot be used with non-stationary
variables, which is a common feature of financial assets. VECM is a linear model
but in terms of variable differences. If cointegration exists, variable
differences are stationary and they introduce an error correction term which
adjusts coefficients to bring the variables back to equilibrium. In finance,
many economic time series turn to be stationary when they are differentiated and
cointegration restrictions often improves forecasting~\cite{duy1998}. Therefore,
VECM has been widely adopted.

Both VECM and VAR model parameters are obtained using ordinary least squares
(OLS) method. OLS has two main problems: is sensitive to errors on input data
and involves many calculations. The former problem is commonly solved using
Ridge Regression (RR) \cite{hoerl1970} which introduces a regularization
parameter that leads to an unbiased estimation with better generalization
capability. The second problem of computational complexity depends on the number
of past values and observations considered.  Recently, online learning
algorithms have been proposed to solve problems with large data sets because of
their simplicity and their ability to update the model when new data is
available. 

Our proposal is an online formulation of the VECM called Online VECM (OVECM)
based on consideration of only a sliding window of the historical data.  OVECM
introduces matrix optimizations in order to reduce the number of operations and
also takes into account the fact that cointegration vector space doesn't
experience large changes with small changes in the input data. Moreover, OVECM
uses RR instead of OLS to obtain VECM parameters. Our method is later tested using four
currency rates from the foreign exchange market with a frequency of 10 seconds.
Model efectiveness is focused on out-of-sample forecast rather than in-sample
fitting.  This criteria allows the OVECM prediction capability to be expressed
rather than just explaining data history. Our method performance is compared
with its optimal offline algorithm.


%The next sections are organized as follows: section~\ref{sec:background}
%presents the VAR and VECM, the OVECM algorithm proposed is presented in
%section~\ref{sec:methodology}. Section~\ref{sec:results} gives a description of
%the data used and the tests carried on to show accuracy and time comparison of
%our proposal against the traditional VECM and
%section~\ref{sec:conclusions} includes conclusions and a proposal for future
%tudy.


\section{\uppercase{Methodology}}
\label{sec:methodology}

Since financial problems are stream data problems, it is unfeasible to include
all input data in a VECM model. Our proposal consists on an online version of
VECM (OVECM) capable of updating the model with new arrival data and give
responses in a short period of time. OVECM only considers a time varying window
of historical data and the using of RR or its variant AAR (Aggregating Algorithm
for Regression) as an alternative of OLS to get model parameters.

On the other hand, getting cointegration vectors using Johansen method is also
an expensive procedure. However, since cointegration
vectors represent the long-run relationship between the time series, they
vary little in time. Our proposal obtains new cointegration vectors only when
this long-term relationship changes. This change is detected by tracking 
the Mean Absolute Percentage Error (MAPE) of the last $n$ in-sample forecasts.

%Since VECM is a model based on time series differences, the MAPE is obtained
%from $\Delta \mathbf{y}$ as following:
% 
%\begin{equation}\label{eq:MAPE}
%\text{MAPE}[t] = \frac{1}{n} \sum_{i=1}^{n} \left| 
%\frac{\Delta \mathbf{y}_{\text{true}}[t-i]-\Delta
%\mathbf{y}_{\text{pred}}[t-i]}{\Delta \mathbf{y}_{\text{true}[t-i]}}
%\right| \, , 
%\end{equation}
%
%\noindent where $\Delta \mathbf{y}_{\text{true}}$ is the actual value of the
%time series $\mathbf{y}$ differences and $\Delta \mathbf{y}_{\text{pred}}$ is
%their forecast value.




%The method proposed consists of
%a modification of RR considering a sliding window which contains only
%the last $L$ samples %and the new input $\mathbf{x}_t$, 
%i.e. $\{\mathbf{x}_i\}_{i=t-L+1}^{t-1}$. 

VECM($p$) for $l$ cointegrated variables has the following form:

\begin{equation}\label{eq:vecfull}
\Delta\mathbf{y}_t 
= \boldsymbol{\alpha\beta}^\top\mathbf{y}_{t-1} 
  + \sum_{i=1}^{p-1}\boldsymbol{\Phi}_i^*\,\Delta\mathbf{y}_{t-i}
  + \mathbf{c} + \boldsymbol{\epsilon}_t \qquad \forall t = p+1,\dots,N.
\end{equation}

Transposing each equation of the system (\ref{eq:vecfull}) we can write
the VECM($p$) model in block-matrix form as:
\begin{equation}\label{eq:vareq}
\mathbf{B} = 
\mathbf{A} \mathbf{X} + 
\mathbf{E} \, , 
\end{equation}
%
\noindent where $\mathbf{B}$ dimension is $((N-p)\times l)$, $\mathbf{A}$
dimension is $((N-p)\times(r+(p-1)l +1))$, $\mathbf{X}$ dimension is $((r+(p-1)l
+1)\times l)$, $\mathbf{E}$ dimension is $((N-p)\times l)$ and $r$ is the number
of cointegration vectors:
%
\begin{alignat}{3}
\mathbf{B}
&= \begin{bmatrix}
   \Delta\mathbf{y}_{p+1}^\top \\
   \Delta\mathbf{y}_{p+2}^\top \\
   \vdots \\
   \Delta\mathbf{y}_N^\top
   \end{bmatrix}
&\quad
\mathbf{X}
&= \begin{bmatrix}
   \boldsymbol{\alpha}^\top \\
   \boldsymbol{\Phi}_1^{*\top} \\
   \boldsymbol{\Phi}_2^{*\top} \\
   \vdots \\
   \boldsymbol{\Phi}_{p-1}^{*\top} \\
   \mathbf{c}^\top
   \end{bmatrix}
&\quad
\mathbf{E}
&= \begin{bmatrix}
   \boldsymbol{\epsilon}_{p+1}^\top \\
   \boldsymbol{\epsilon}_{p+2}^\top \\
   \vdots \\
   \boldsymbol{\epsilon}_N^\top \\
   \end{bmatrix}
\end{alignat}
\noindent and 
\begin{align}
\mathbf{A} 
&= \begin{pmat}[{....|}]
   \mathbf{y}_p^\top \boldsymbol{\beta} & \Delta \mathbf{y}_p^\top & \Delta\mathbf{y}_{p-1}^\top & \dots 
                    & \Delta\mathbf{y}_2^\top & 1 \cr
   \mathbf{y}_{p+1}^\top  \boldsymbol{\beta} &\Delta\mathbf{y}_{p+1}^\top & \Delta\mathbf{y}_p^\top & \dots
                       & \Delta\mathbf{y}_3^\top & 1 \cr
   \vdots & \vdots & \vdots & \ddots & \vdots & \vdots \cr
   \mathbf{y}_{N-1}^\top  \boldsymbol{\beta} &\Delta\mathbf{y}_{N-1}^\top & \Delta\mathbf{y}_{N-2}^\top & \dots 
                       & \Delta\mathbf{y}_{N-p-1}^\top & 1 \cr
   \end{pmat}\, .
\end{align}
%Taking into account the error term $\mathbf{E}$, equation~(\ref{eq:vareq}) 
%can be solved with respect to $\mathbf{X}$ using the ordinary least
%squares estimation.
However, the number of rows of matrix $\mathbf{A}$ increases with the number of
observations $\mathbf{y}_t$. Our proposal considers only the most recent $L$
rows of matrices $\mathbf{A}$ and $\mathbf{B}$ defined as  $\mathbf{A}(t)$ and
$\mathbf{B}(t)$:

\begin{equation}
\label{eq:notation}
	\mathbf{A}(t) = 
\left[
  \begin{tabular}{c>{$}c<{$}c}
    --- & \mathbf{a}^{\top}_{t-L} & ---\\
    --- & \mathbf{a}^{\top}_{t-L+1} & ---\\
    & \vdots & \\
    --- & \mathbf{a}^{\top}_{t} & ---
  \end{tabular}
\right]
\quad \text{and} \quad
\mathbf{B}(t) =
\left[
  \begin{tabular}{c>{$}c<{$}c}
    --- & \mathbf{b}_{t-L} & ---\\
    --- & \mathbf{b}_{t-L+1} & ---\\
    & \vdots & \\
    --- & \mathbf{b}_{t} & ---
  \end{tabular}
\right] \, ,
\
\end{equation}

\noindent so that VECM parameters $\mathbf{X}(t)$ are found using the following
equation:

\begin{equation}
\mathbf{B}(t) = \mathbf{A}(t) \mathbf{X}(t) + \mathbf{E}(t)\\
\end{equation}

The RR solution $\mathbf{\hat{X}}(t)$ using the sliding window matrices defined above
is:

\begin{equation}
\label{eq:oproblem}
\mathbf{\hat{X}}(t)=\mathbf{S}(t)^{-1} \mathbf{W}(t) \, ,
\end{equation}

\noindent where $\mathbf{S}{t}$ and $\mathbf{W}{t}$ are define as:

\begin{eqnarray*}
\mathbf{S}(t) =&{\bf A}(t)^\top{\bf A}(t)+ \lambda \mathbb{I} &=
\sum_{i=0}^L \mathbf{a}_{t-i}\mathbf{a}_{t-i}^\top + \lambda \mathbb{I}
\label{eq:S} \\
\mathbf{W}(t) =& {\bf A}(t)^\top{\bf B}(t) &= 
\sum_{i=0}^L \mathbf{a}_{t-i}\mathbf{b}_{t-i} 
\label{eq:W}
\end{eqnarray*}

It is worth noticing that, at the next time step, the matrices $\mathbf{S}(t+1)$
and $\mathbf{W}(t+1)$ are slightly different to $\mathbf{S}(t)$ and
$\mathbf{W}(t)$:

\begin{eqnarray*}
\mathbf{S}(t+1)&=&
\mathbf{S}(t) +
\mathbf{a}_{t+1}
\mathbf{a}_{t+1}^\top -
\mathbf{a}_{t-L} \mathbf{a}_{t-L}^\top \\
\mathbf{W}(t+1)&=&
\mathbf{W}(t) +
\mathbf{a}_{t+1}
\mathbf{b}_{t+1} -
\mathbf{a}_{t-L} \mathbf{b}_{t-L} \, .
\end{eqnarray*}

%Agregar solo si se decide agregar Sherman-Morrison
%Therefore the Sherman-Morrison-Woodbury formula can be used to reduce inverse matrix
%calculations.

The algorithm~\ref{alg:proposal} shows OVECM using three different
methods for geting model parameters: OLS, RR, AAR which leads to three different
algorithms OVECM-OLS, OVECM-RR, OVECM-AAR:

\begin{algorithm}[ht]
\begin{algorithmic}[1]
\REQUIRE $\,$ \\
$\mathbf{y}$: matrix with $N$ input vectors and $l$ time series\\
$p$: number of past values \\
method: \{'OLS','RR','AAR'\} \\
$\lambda$: regularization parameter (only requires for RR and AAR methods) \\
$L$: window size of data($L<N$) \\
$\text{mean\_error}$: MAPE threshold \\
$n$: windows size to obtain in-sample MAPE \\
\ENSURE  $\,$ \\
$\{\mathbf{y}_{\text{pred}}[L+1],\dots, \mathbf{y}_{\text{pred}}[N]\}$: model predictions 
\STATE solver = new \texttt{Solver}(method,$\lambda$) \\
\FOR { $t =1$ to $N-L$ }
    \STATE $\mathbf{Y} \gets \mathbf{y}[t:t+L-1]$
    \STATE $\mathbf{y}_t \gets \mathbf{y}[t+L]$
	\IF {$t = 1$}
	    \STATE{$v \gets \texttt{getJohansen}(\mathbf{Y},p)$}
	    \STATE{$[\mathbf{A} \quad \mathbf{B}] \gets
        \texttt{vecMatrix}(\mathbf{Y},p,v)$}
        \STATE \texttt{solver.init\_model($\mathbf{A},\mathbf{B}$)} 
    \ENDIF
	\STATE{[$\mathbf{A} \quad \mathbf{B} \quad \mathbf{a}_t \quad \mathbf{a}_L \quad \mathbf{b}_t \quad
    \mathbf{b}_L ] \gets
    \texttt{vecMatrixOnline}(\mathbf{Y},\mathbf{y}_t,p,v)$}
    \STATE $[\mathbf{X} \quad \mathbf{y}_{\text{pred}}[t]] \gets \texttt{solver.regression}
    (\mathbf{A},\mathbf{B},\mathbf{a}_t,\mathbf{b}_t)$
    \STATE $\mathbf{Y}_{\text{pred}} = \mathbf{AX}[-n:]+\mathbf{Y}[-n-1:-1]$
    \STATE $\mathbf{e} = \texttt{mape}(\mathbf{Y}_{\text{pred}},\mathbf{Y}[-n:])$
    \IF {$\texttt{mean}(\mathbf{e}) > \text{mean\_error}$}
	    \STATE{$v \gets \texttt{getJohansen}(\mathbf{Y},p)$}
	    \STATE{$\mathbf{A} \gets
        \texttt{vecMatrixUpdate}(\mathbf{A},\mathbf{Y},p,v)$}
        \STATE \texttt{solver.init\_model($\mathbf{A},\mathbf{B}$)} 
	    \STATE{$[\mathbf{a}_t \quad \mathbf{b}_t] \gets
        \texttt{vecMatrixOnline}(\mathbf{Y},\mathbf{y}_t,p,v)$}
        \STATE $[\mathbf{X} \quad \mathbf{y}_{\text{pred}}[t]] \gets \texttt{solver.regression}
        (\mathbf{A},\mathbf{B},\mathbf{a}_t,\mathbf{b}_t)$
    \ENDIF
\STATE{$[\mathbf{A} \quad \mathbf{B}] \gets
\texttt{matrixUpdate}(\mathbf{A},\mathbf{B},\mathbf{a}_t,\mathbf{b}t)$}
\ENDFOR
\end{algorithmic}
\caption{OVECM: Online VECM}
\label{alg:proposal}
\end{algorithm}


Our proposal considers the following:

\begin{itemize}
\item The function \texttt{getJohansen} returns cointegration vectors given by the
Johansen method considering the trace statistic test at 95\% level of
significance. This procedure is called at the first step of the algorithm and
every time in-sample MAPE ($\mathbf{e}$) exceeds threshold error defined (mean\_error).
\item The function \texttt{vecMatrix} returns VECM
matrices $\mathbf{A}(t),\mathbf{B}(t)$ shown in equation~(\ref{eq:notation}). This
method is only required at the first step.
\item The function \texttt{vecMatrixOnline} returns new rows $\mathbf{a}_t^\top$ and
$\mathbf{b}_t^\top$ given new input data $\mathbf{y}_t$.
\item The Solver class (see algorithm \ref{alg:solver}) obtain VECM parameters  
using RR, AAR or OLS methods.
\item The function \texttt{vecMatrixUpdate} updates matrix $\mathbf{A}$ when new
cointegration vectors are required (matrix $\mathbf{B}$ is not affected). Model
parameters $\mathbf{X}$ is also updated later.
%\item The number of cointegration vectors is set as a parameter (aun no lo
%coloco en el algoritmo).
%\item In order to set VECM parameter: $L$ and $p$, we use the Akaike Information
%Criterion (AIC). RR parameter $\lambda$ was done by cross-validation.
%\item Obtain VECM parameter $\mathbf{X}(t)$ in equation \ref{eq:optsolSLAAR} requires
%to calculate the inverse of matrix $\mathbf{S}(t+1)$ which is an expensive
%routine.However, since we already know $\mathbf{S}(t)^{-1}$ we can use
%Sherman-Morrison-Woodbury twice~\ref{eq:SMW}.
\end{itemize}




\begin{algorithm}[ht]
\begin{algorithmic}[1]
\REQUIRE $\,$ \\
method: \{'OLS','RR','AAR'\} \\
$\mathbf{A}$: VECM design matrix \\
$\mathbf{B}$: VECM dependant variables \\
$\mathbf{a}_t$: new row of $\mathbf{A}$ \\
$\mathbf{b}_t$: new row of $\mathbf{B}$ \\
$\gamma$: regularization parameter \\
$n$: windows size to obtain in-sample MAPE \\
\ENSURE  $\,$ \\
$\mathbf{X}$: regression solution \\
$\mathbf{e}$: in-sample MAPE \\
%\quad \\
%\texttt{init}($L,\lambda$)
%\STATE self.L = L
%\STATE self.lambda = $\lambda$
\quad \\
\texttt{init\_model}($\mathbf{A},\mathbf{B}$)\\
\IF {method == 'RR' or method == 'AAR'}
\STATE $ [m \quad n] = \text{size}(\mathbf{A}) $ 
\STATE $\mathbf{S} = \displaystyle \sum_{i=1}^m \mathbf{a}_i \mathbf{a}_i^\top + \gamma \mathbb{I}$
\STATE $\mathbf{W} = \displaystyle \sum_{i=1}^m \mathbf{a}_i \mathbf{b}_i$
\ENDIF \\
\quad \\
\texttt{regression}($\mathbf{A},\mathbf{B},\mathbf{a_t},\mathbf{b_t},n$) \\
\IF {method == 'OLS'}
        \STATE $\mathbf{X}=(\mathbf{A}^\top \mathbf{A})^{-1}\mathbf{A}^\top
        \mathbf{B}$
\ELSIF {method == 'RR'}
        \STATE $\mathbf{X} = \mathbf{S}^{-1} \mathbf{W} $
        \STATE $\mathbf{S} = \mathbf{S}+
        \mathbf{a}_t \mathbf{a}_t^\top-
        \mathbf{a}_1 \mathbf{a}_1^\top$
        \STATE $\mathbf{W} = \mathbf{W} + \mathbf{a}_t \mathbf{b}_t$
\ELSIF {method == 'ARR'}
        \STATE $\mathbf{S} = \mathbf{S}+
        \mathbf{a}_t \mathbf{a}_t^\top-
        \mathbf{a}_1 \mathbf{a}_1^\top$
        \STATE $\mathbf{X} = \mathbf{S}^{-1} \mathbf{W} $
        \STATE $\mathbf{W} = \mathbf{W} + \mathbf{a}_t \mathbf{b}_t$
\ENDIF
\STATE $\mathbf{y}_{\text{pred}} = \mathbf{X}^\top \mathbf{a}_t$
\end{algorithmic}
\caption{Solver Class for Regression Methods}
\label{alg:solver}
\end{algorithm}




% Aqui se puede hablar de evaluation using competitive analysis. The idea
%of competitiveness is to compare the output generated by an online algorithm to
%the output produced by an optimal offline algorithm. An optimal online algorithm
%is an omniscient algorithm that knows the entire input data in advance and can
%compute an optimal output. The better an online algorithm approximates the
%optimal solution, the more competitive this algorithm is.


The proposal was compared against an optimal online algorithm which is a time varying
VECM (OOVECM). Algorithm ~\ref{alg:OOVECM} shows this modification:


\begin{algorithm}[ht]
\begin{algorithmic}[1]
\REQUIRE $\,$ \\
$\mathbf{y}$: matrix with $N$ input vectors and $l$ time series\\
$p$: number of past values \\
$L$: sliding window size ($L<N$) \\
\ENSURE  $\,$ \\
$\{\Delta \mathbf{y}_{\text{pred}}[L+1],\dots,\Delta \mathbf{y}_{\text{pred}}[N]\}$: model predictions 
\FOR { $i =0$ to $N-L$ }
    \STATE $\mathbf{y}_i \gets \mathbf{y}[i:i+L]$
	\STATE{$v \gets \texttt{getJohansen}(\mathbf{y}_i,p)$}
	\STATE{$[\mathbf{A} \quad \mathbf{B}] \gets
    \texttt{vecMatrix}(\mathbf{y}_i,p,v)$}
    \STATE $\mathbf{X} \gets \text{OLS} (\mathbf{A},\mathbf{B})$%\mathbf{(A^\top A)^{-1}A^\top B}$
    \STATE $\Delta \mathbf{Y}_{\text{true}}[i] \gets \mathbf{B}[-1,:]$
    \STATE $\Delta \mathbf{Y}_{\text{pred}}[i] \gets \mathbf{A}[-1,:] \times \mathbf{X}$
\ENDFOR
    \STATE $\text{MAPE} \gets \texttt{mape}(\Delta \mathbf{Y}_{\text{true}}, \Delta
    \mathbf{Y}_{\text{pred}})$
\end{algorithmic}
\caption{OOVECM: Optimal Online Vector Error Correction Model}
\label{alg:OOVECM}
\end{algorithm}


%%Solo para tesis
%\section{Online VEC model}
%
%If we want to update the VEC model with new input data
%$\mathbf{y}_{T+1}$, we have to add a new row in matrices
%$\mathbf{Y,X}$ and $\mathbf{E}$. 
%
%\begin{eqnarray}
%\mathbf{Y} &=& 
%                \left[ \begin{array}{ccc}
%               \quad & \mathbf{\Delta y}_{p+1} & \quad \\ \hline
%               \quad & \mathbf{\Delta y}_{p+2} & \quad \\ \hline
%               \quad & \vdots & \quad \\ \hline 
%               \quad & \mathbf{\Delta y}_T & \quad \\ 
%               \quad & \color{red}\mathbf{\Delta y}_{T+1} & \quad 
%               \end{array} \right]
%             \\
%\mathbf{\Phi^*} &=& 
%                \left[ \begin{array}{ccc}
%               \phi_{p-1}^* \\ 
%               \vdots \\ 
%               \phi_{1}^* \\
%               \alpha \\
%                c   
%               \end{array} \right]
%\\
%\mathbf{X} &=& \begin{bmatrix} \label{offX}
%   \mathbf{\Delta y}_2 & \dots & \mathbf{\Delta y}_{p-1} &
%   \mathbf{\Delta y}_{p} & \beta'\mathbf{y}_{p} & 1\\
%   \mathbf{\Delta y}_3 & \dots & \mathbf{\Delta y}_{p} &
%   \mathbf{\Delta y}_{p+1} & \beta'\mathbf{y}_{p+1} &1\\
%   \vdots &  \ddots & \vdots & \vdots & \vdots & \vdots\\
%   \mathbf{\Delta y}_{T-p+1} & \dots & \mathbf{\Delta y}_{T-2} &
%   \mathbf{\Delta y}_{T-1} & \beta'\mathbf{y}_{T-1} &1 \\
%   \color{red} \mathbf{\Delta y}_{T-p+2} & \dots & \color{red}\mathbf{\Delta y}_{T-1} &
%   \color{red}\mathbf{\Delta y}_{T} & \color{red}\beta'\mathbf{y}_{T}
%   & \color{red} 1
%   \end{bmatrix}
%\\
%\mathbf{E} &=& \begin{bmatrix}
%              \mathbf{\epsilon}_{p+1} \\ \vdots \\ \mathbf{\epsilon}_T
%              \\ \color{red}\mathbf{\epsilon}_{T+1}
%             \end{bmatrix}
%\end{eqnarray}

