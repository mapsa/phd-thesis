
\chapter{Conclusions and Future Work}
\label{chapter:conclusions}
Main results and contributions are presented in this chapter along with some
suggested future work.

\vspace{0.5cm} 

\section{Conclusions of this thesis}

This thesis is a multidisciplinary work that involves knowledge of Finance
and Economics, Time Series, Machine Learning, Parallel Computing and Scientific
Computing. We specifically addressed the joint dynamic behaviour of financial time
series that are said to be cointegrated. A broadly used approach, called VECM (Vector
Error Correction Model), characterises this behaviour in terms of forces pulling
towards equilibrium called cointegration. However, the use of VECM has been
limited to low frequency time series processed in batch mode. In this thesis we
explored the use of VECM with high frequency data and we found that the main
limitation was computational. The focus of our work was developing a model to improve the performance forecasting of financial time series maintaining good execution times in order to use it with high frequency data. VECM parameter estimation uses two computationally
expensive routines: the Johansen method, to obtain cointegration vectors, and
the ordinary least squares method to solve the system. In this thesis, different
ways to explore cointegration in high frequency time series were proposed,
always considering the computational limitations of the VECM.

The study of cointegration in high frequency data was done using two different
approaches presented in Chapters \ref{chapter:proposal1} and
\ref{chapter:proposal2}, called AVECM and OVECM:
\begin{description}
\item[AVECM] AVECM is an adaptive version of VECM which includes a new method to choose VECM parameters based on
the maximisation of the percentage of cointegration. AVECM was implemented using
MPI to search on a grid of possible values. We evaluated our results in terms of performance and execution times:
\begin{description}
\item[Performance] Results showed that AVECM improves
performance measures by finding parameters of L and p maximising the percentage
of cointegration. AVECM performance was compared against ARIMA and the random walk model. We showed that the out-of-sample performance for AVECM is superior to ARIMA and the naive random walk model in terms of the MSE and $U$-statistic. This result is very important since it is still difficult to outperform ARIMA and the random walk model for standard econometric forecasting models  despite their simplicity. 
\item[Execution times] execution times were reduced more than 9 times
ensuring a response time before the processing of the next data point. Despite the fact that high frequency Forex data can be spurious, the model
performance can be less reliable (and more spurious) relative to the lower
frequencies (such as 1 minute or 5 minute intervals) adopted by some other
studies. However, the deficiency is offset by gain in accuracy from parallel
processing which is capable of searching or examining a much larger state space
given the same computational time.
\end{description}
\item[OVECM] We proposed an online version of VECM (OVECM). OVECM optimises how
model parameters are obtained using a sliding window of the most recent data.
OVECM, unlike AVECM, updates the parameters at each step, instead of obtaining
new ones. This proposal takes advantage of the long-run relationship
between the time series in order to obtain improved execution times. OVECM also
introduces matrix optimisation in order to obtain the new model in an iterative
way.  
\begin{description}
\item[Performance] 
OVECM was compared with the traditional VECM and ARIMA. Despite the fact that traditional VECM slightly outperformed our proposal, the OVECM execution time is lower. This result can be explained because OVECM avoids the calculation of cointegration vectors at each time step which improves execution times but can affect forecast performance.
\item[Execution times] OVECM was compared with the traditional VECM and ARIMA
with the same sliding window sizes. As a result, OVECM outperformed both in terms of
execution time. This reduction of execution time is mainly because OVECM avoids the cointegration vector calculation using the Johansen method. Aditionally, OVECM took much less than a minute at every
step making it possible to use with higher frequency data.  
\end{description}
\end{description}
Finally, cointegration information can now easily be used as an integration tool
to detect arbitrage opportunities or risk control in financial time series.

\section{Contributions of this thesis}

This research helped to integrate machine learning techniques and parallel computing with econometric models such as VECM so they can be used with high frequency data. We also observed that cointegration relationships are sensitive to the choice of the amount of data and VECM parameters and this can be used to improve forecasting performance. We proposed two variations of VECM, OVECM and AVECM which could be extended to other econometric models.

Regarding the initial objectives defined at the beginning of this thesis we can say:


\begin{itemize}
\item $\Large \mathcal{O}_1$: \emph{A review of the literature on time series
analysis models including machine learning techniques.}\\
The literature review was focused in finance, time series concepts and models and machine learning including theory and applications.
\item $\Large \mathcal{O}_2$: \emph{Development of a set of known features of the
studied time series and the application to improve forecasting.} \\
We included specific knowledge about forex rates, which were those used in our experiments, such as trade best trading times, stylised facts, percentage of cointegration, time series frequency, among others.
\item $\Large \mathcal{O}_3$: \emph{Development of parallel and efficient
algorithms to ensure quick response times .} \\
We showed that high performance computing will allow the use of computationally expensive methods to forecast high frequency data ensuring quick response times.
\item $\Large \mathcal{O}_4$: \emph{Deep mathematical analysis of the proposal
and financial concepts involved.} \\
The two proposals were presented with a strong mathematical foundation, we added definitions and proofs of all the concepts involved.
\item $\Large \mathcal{O}_5$: \emph{Design and implement representative set of
experiments in order to show when and why the proposal performs better.}
All the experiments were designed to give a general view of the method performance. We clearly showed the use and limitations of our proposals.
\end{itemize}


\section{Future Work}

For future study, it would be interesting to explore the relationship between
cointegration and performance for more assets, including not only forex rates
but also stocks. It would also be interesting to include more explaining
variables such as bid-ask spread and change in volume.

The online approach for other econometrics models it will be also worthy of 
study. Many of them could be adapted and used with higher frequency data. In
this thesis, the online version of OLS was studied, but the online version of
ridge regression could also be applied to obtain a solution with better
generalisation capabilities.

Financial time series are specially known for presenting high heteroscedasticity, common shocks in volatility could be wrongly interpreted as cointegration. To tackle this problem, a second step of the adaptive VECM algorithm could be implemented to the residuals. A good option could be a GARCH model, for example.  

Finally, it will be also interesting to improve the out-of-sample forecast
by considering more explicative variables, to increase window sizes or try different
conditions to obtain new cointegration vectors.

