
\chapter{Conclusions and Future Work}
\label{chapter:conclusions}
Main results and contributions are presented in this chapter along with some
suggested future work.

\vspace{0.5cm} 

\section{Conclusions of this thesis}

This thesis is a multidisciplinary work that involves knowledge of Finance
and Economics, Time Series, Machine Learning, Parallel Computing and Scientific
Computing. We specifically addressed the joint dynamic behaviour of financial time
series that said to be cointegrated. A broadly used approach, called VECM (Vector
Error Correction Model), characterises this behaviour in terms of forces pulling
towards equilibrium called cointegration. However, the use of VECM has been
limited to low frequency time series processed in batch mode. In this thesis we
explored the use of VECM with high frequency data and we found that the main
limitation was computational. VECM parameter estimation uses two computationally
expensive routines: the Johansen method, to obtain cointegration vectors, and
the ordinary least squares method to solve the system. In this thesis, different
ways to explore cointegration in high frequency time series were proposed,
always considering the computational limitations of the VECM.

The study of cointegration in high frequency data was done using two different
approaches presented in Chapter \ref{chapter:proposal1} and
\ref{chapter:proposal2}. First, we used a parallel version of VECM, called AVECM
(Adaptive VECM), which includes a new method to choose VECM parameters based on
the maximisation of the percentage of cointegration. AVECM was implemented using
MPI to search on a grid of possible values. Results showed that AVECM improves
performance measures by finding parameters of L and p maximising the percentage
of cointegration and also execution times were reduced more than 9 times
ensuring a response time before the processing of the next data point.
Secondly, we proposed and online version of VECM (OVECM). OVECM optimises how
model parameters are obtained using a sliding window of the most recent data.
OVECM, unlike AVECM, updates the parameters at each step, instead of obtaining
new ones. This proposal also takes advantage of the long-run relationship
between the time series in order to obtain improved execution times. OVECM also
introduces matrix optimisation in order to get the new model in an iterative
way.  As a result, OVECM took much less than a minute at every
step making it possible to use with higher frequency data.  

\section{Contributions of this thesis}


Based on the observation that cointegration
relationships are sensitive to the choose of the amount of data and VECM
parameters,

\section{Future Work}

For future study, it would be interesting to explore the relationship between
cointegration and performance for more assets, including not only forex rates
but also stocks. It would also be interesting to include more explaining
variables such as bid-ask spread and change in volume.

The online approach for other econometrics models it will be also worth of
study, many of the could be adapted and be used with higher frequency data. In
this thesis, the online version of OLS was studied, but the online version of
ridge regression could also be applied to obtain a solution with better
generalisation capabilities.

Finally, it will be also interesting to improve the out-of-sample forecast
by considering more explicative variables, to increase window sizes or trying new
conditions to obtain new cointegration vectors.

