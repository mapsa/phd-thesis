\chapter{Introduction}\label{ch:introduction}

\vspace{0.5cm} 

"An individual economic variable, viewed as a time series, can wander extensively
and yet some pairs of series may be expected to move so that they do not drift
far apart."-Robert F. Engle and Clive W.J. Granger~\cite{engle1987}.\\
In this introduction we present the scope, objectives, hypothesis and
organization of this thesis.


\section{Scope of this research}

The stochastic behaviour of financial time series, its incrementing
amount of data available and the need of performing accurate forecasting in
short periods of time has motivated researchers to create efficient and fast
algorithms. This study involves interdisciplinary knowledge such as: finance,
scientific computing, high performance computing, machine learning
among others.

Forecasting financial time series have been modelled using classical statistical
approaches. More recently, machine learning models have been extensively used in
forecasting. However, their main disadvantage is that getting model parameters
is a computational challenge.  The computational complexity of machine learning
algorithms has become a limiting factor for problems that require processing
large volumes of data and where response time is crucial.

Therefore, algorithms that process large amount of data in a short periods of
time are required. Recently, online learning algorithms have been developed to solve
large-scale problems since they process an instance at a time, updating the
model at each step incrementally. This is opposed to the batch algorithms where
the forecast model is built using a large collection of historical data in
a training phase.

The specific scope of this study is to use financial time series features in
order to design a forecasting algorithm which ensures accuracy and low response
times. Cointegration is the main feature studied and it refers that one or more
linear combinations of these time series are stationary even though individually
they are not~\cite{engle87}.  Some models, such as the Vector Error Correction Model
(VECM), take advantage of this property and describe the joint behaviour of
several cointegrated variables.

%%
In this thesis, an online formulation of the VECM called Online VECM (OVECM) is
proposed. OVECM is based on the consideration of a sliding window of the most
recent data.  The algorithm introduces matrix optimisations in order to reduce
the number of operations and also takes into account the fact that cointegration
vector space doesn't experience large changes with small changes in the input
data. Moreover, VECM parameters are obtained using machine learning methods. Our
method is later tested using four currency rates from the foreign exchange
market with different frequencies.  On the other hand, in order to improve VECM
parameters, an adaptive VECM algorithm is presented called AVECM. AVECM allows
VECM parameters to be found by maximising the number of cointegration relations
for a given set of parameters on a grid search. This grid search is done in
parallel.

Models effectiveness were focused on the out-of-sample forecast rather than on
the in-sample fitting. This criteria allows the OVECM and AVECM prediction
capability to be expressed rather than just explaining data history. Our method
performance is compared with the naive forecast of the random walk model and
ARIMA which are the most widely used algorithms for modelling a multivariate
time series.


\section{Research Objectives}
The main motivation for this research is the development of efficient methods
for forecasting financial time series.

The specific objectives of this research are,
\begin{itemize}
\item $\Large \mathcal{O}_1$: \emph{A review of the literature on time series
analysis models including machine learning techniques.}
\item $\Large \mathcal{O}_2$: \emph{Development of a set of known features of the
studied time series and the application to improve forecasting.}
\item $\Large \mathcal{O}_3$: \emph{Development of parallel and efficient
algorithms to ensure quick response times .}
\item $\Large \mathcal{O}_4$: \emph{Deep mathematical analysis of the proposal
and financial concepts involved.}
\item $\Large \mathcal{O}_5$: \emph{Design and implement representative set of
experiments in order to show when and why the proposal performs better.}
\end{itemize}


\section{Research Hypothesis}


Cointegration concept was introduced by Engle and Granger in 1987
\cite{engle1987} and implies that one or more linear combinations of
non-stationary variables are stationary even though individually they are not.
Moreover Stock and Watson in 1988 \cite{stock+watson1988} observed that
cointegration reflects the common stochastic trends providing a useful way to
understand cointegration relationships. These common stochastic trends can be
also interpreted as a long-run equilibrium relationships.

Vector error correction model (VECM) introduces this long-run relationship
among a set of cointegrated variables as an error correction term. VAR model
expresses future values as a linear combination of variables past values.
However, VAR model cannot be used with non-stationary variables. VECM is a
linear model but in terms of variable differences. If cointegration exists,
variable differences are stationary and they introduce an error correction term
which adjusts coefficients to bring the variables back to equilibrium. In
finance, many economic time series turn to be stationary when they are
differentiated and cointegration restrictions often improves
forecasting~\cite{duy1998}. Therefore, VECM has been widely adopted.

Both VECM and VAR model parameters are obtained using ordinary least squares
(OLS) method. OLS has two main problems: is sensitive to errors on input data
and involves many calculations. The former problem is commonly solved using
Ridge Regression (RR) \cite{hoerl1970} which introduces a regularization
parameter that leads to an unbiased estimation with better generalisation
capability. The second problem of computational complexity depends on the number
of past values and observations considered.  Recently, online learning
algorithms have been proposed to solve problems with large data sets because of
their simplicity and their ability to update the model when new data is
available. 

The main research hypothesis explored in this dissertation is the following:
\\

\textit{An online learning algorithms based on cointegration and high
performance computing will allow faster forecasting
algorithms for financial time series to be obtained while maintaining good accuracy levels.}


\section{Organisation of this Thesis}

Chapter \ref{chapter:HFT} contains relevant finance concepts required to understand financial
time series models such as market hypothesis, frequency, order book generation,
market microstructure.
Chapter \ref{chapter:FTT} discuss main characteristics and concepts of financial
time series including classic models such as ARMA, ARIMA,
GARCH and more recent ones such as VECM, VAR and volatility models.
Chapter \ref{chapter:MLM} gives an introduction to machine learning and its
variant online learning. 
Chapter \ref{chapter:proposal1} presents the first proposal called AVECM which
includes a new parallel method to choose VECM parameters based on the maximisation of the
percentage of cointegration.
In Chapter \ref{chapter:proposal2} a second approach is presented called OVECM,
which is an online version of OVECM for high frequency data.
Chapter \ref{chapter:conclusions} presents a discussion of found results and summarise the main
conclusions of this thesis. It also presents some research directions for future
study.


