\chapter{Proofs}
Some useful proofs for the understanding of this thesis.

\section{Pseudo-inverse computed using the compact SVD}\label{app:pscsvd}

\textbf{Proof}\quad

Since $\mathbf{A}$ is singular the problem shown in
equation~(\ref{eq:regressionproblem}) has not solution, the minimum norm given
by equation~(\ref{eq:MP}) is obtained by solving the equivalent problem:

\begin{equation*}
\label{eq:proyectorsol}
\mathbf{A \hat{X} = PB} 
\end{equation*}

\noindent where $\mathbf{P=U_1 U_1^\top}$ is the projection onto the
Col($\mathbf{A}$). 

Since $\mathbf{V} = [\underset{(n \times k)}{\mathbf{V_1}} |
\underset{(n \times k)}{\mathbf{V_2}}]$ and $\mathbf{V_1^\top V_2 =
0}$ we can express $\mathbf{\hat{X}} = \mathbf{V_1 x_1 + V_2 x_2}$
with $\mathbf{x_2=0}$ because $\mathbf{\hat{X}}$ lives in the
$\text{Row}(\mathbf{A})$ given by $\mathbf{V_1}$, so we have:

\begin{eqnarray*}
\mathbf{A \hat{X}} &=& \mathbf{PB} \\
\mathbf{U_1 \Sigma_1 V_1^\top \hat{X}} &=& \mathbf{U_1 U_1^\top B} \\
\mathbf{ V_1^\top \hat{X}} &=&  \mathbf{\Sigma_1^{-1} U_1^\top B} \\ 
\mathbf{ V_1^\top V_1 x_1} &=& \mathbf{\Sigma_1^{-1}
U_1^\top B} \\
\mathbf{x_1}&=& \mathbf{\Sigma_1^{-1} U_1^\top B}
\end{eqnarray*}

\noindent from this result we can obtain $\mathbf{\hat{X}}$ and
therefore the pseudo-inverse expression:

\begin{eqnarray*}
\mathbf{\hat{X}} &=& \mathbf{V_1 x_1} \\
                &=& \mathbf{V_1 \Sigma_1^{-1} U_1^\top B} \\
\mathbf{A^+} &=& \mathbf{V_1 \Sigma_1^{-1} U_1^\top} \, .
\end{eqnarray*}

$\blacksquare$

\section{The pseudo-inverse computed using the compact
singular value decomposition (SVD)} \label{app:pseudoproof}
\begin{equation}
\mathbf{A}^+ = \mathbf{V_1\Sigma_1^{-1}U_1^\top}
\end{equation}

\textbf{Proof}\quad

In the case matrix $\mathbf{A}$ is non singular, its solution is
straight forward:

\begin{equation}
\label{OLSsolution2}
    \mathbf{\mathbf{X}}=\mathbf{A}^{-1}\mathbf{Y}
\end{equation}


Since the problem shown in equation~(\ref{eq:regproblem}) has not
solution, the minimum norm given by equation~(\ref{OLSsolution2}) is
obtained by solving the equivalent problem:

\begin{equation*}
\label{eq:proyectorsol}
\mathbf{A \hat{\mathbf{X}} = PY} 
\end{equation*}


\noindent where $\mathbf{P=U_1 U_1^\top}$ is the projection onto the
Col($\mathbf{A}$). 

Since $\mathbf{V} = [\underset{(n \times k)}{\mathbf{V_1}} |
\underset{(n \times k)}{\mathbf{V_2}}]$ and $\mathbf{V_1^\top V_2 =
0}$ we can express $\mathbf{\hat{\mathbf{X}}} = \mathbf{V_1 \mathbf{X}_1 + V_2 \mathbf{X}_2}$
with $\mathbf{\mathbf{X}_2=0}$ because $\mathbf{\hat{\mathbf{X}}}$ lives in the
$\text{Row}(\mathbf{A})$ given by $\mathbf{V_1}$, so we have:

\begin{eqnarray*}
\mathbf{A \hat{\mathbf{X}}} &=& \mathbf{PY} \\
\mathbf{U_1 \Sigma_1 V_1^\top \hat{\mathbf{X}}} &=& \mathbf{U_1 U_1^\top Y} \\
\mathbf{ V_1^\top \hat{\mathbf{X}}} &=&  \mathbf{\Sigma_1^{-1} U_1^\top Y} \\ 
\mathbf{ V_1^\top V_1 \mathbf{X}_1} &=& \mathbf{\Sigma_1^{-1}
U_1^\top Y} \\
\mathbf{\mathbf{X}_1}&=& \mathbf{\Sigma_1^{-1} U_1^\top Y}
\end{eqnarray*}

\noindent from this result we can obtain $\mathbf{\hat{\mathbf{X}}}$ and
therefore the pseudo-inverse expression:

\begin{eqnarray*}
\mathbf{\hat{\mathbf{X}}} &=& \mathbf{V_1 \mathbf{X}_1} \\
                &=& \mathbf{V_1 \Sigma_1^{-1} U_1^\top Y} \\
\mathbf{A^+} &=& \mathbf{V_1 \Sigma_1^{-1} U_1^\top} \, .
\end{eqnarray*}

$\blacksquare$

\section{Ridge regression optimal solution}\label{app:rroptsection}

\begin{equation*}
\mathbf{\mathbf{X}}_*=(\mathbf{A}^\top \mathbf{A}+\lambda \mathbb{I})^{-1}\mathbf{A}^\top y \, ,
\end{equation*}


\textbf{Proof}\quad

\begin{eqnarray*}
J(\mathbf{\mathbf{X}}) &=&  \| \mathbf{A}\mathbf{\mathbf{X}} - \mathbf{Y} \|_2^2  + \lambda
 \| \mathbf{\mathbf{X}}\| ^2 \\
 &=&  \sum_{t=1}^N (\mathbf{\mathbf{X}}^\top {\bf a}_t-y_t)^2 + \lambda \sum_{i=1}^p \mathbf{X}_i^2 \\
 &=& (\mathbf{X}^\top a_1-y_1)^2 + \dots + (\mathbf{X}^\top a_N-y_N)^2 + \lambda (\mathbf{X}_1^2+\dots+\mathbf{X}_p^2)
\end{eqnarray*}
\noindent taking derivatives
 \begin{eqnarray*}
 \frac{\partial J(\mathbf{\mathbf{X}})}{\partial \mathbf{X}_1}&=& 
 2(\mathbf{X}^\top \mathbf{a}_1-y_1)\mathbf{a}_{11} + \dots + 2(\mathbf{X}^\top \mathbf{a}_N-y_N)\mathbf{a}_{N1} + 2\lambda \mathbf{X}_1 \\
 &=& 2\mathbf{a}_1^\top(\mathbf{A}\mathbf{X}-\mathbf{Y}) + 2\lambda\mathbf{X}_1\\
& \vdots &\\
  \frac{\partial J(\mathbf{\mathbf{X}})}{\partial \mathbf{X}_p}&=& 
 2\mathbf{a}_p^\top(\mathbf{A}\mathbf{X}-\mathbf{Y}) + 2\lambda\mathbf{X}_p 
\end{eqnarray*}

Then we have that:
\begin{eqnarray*}
\frac{\partial J(\mathbf{\mathbf{X}})}{\partial \mathbf{X}}&=& 
 2\mathbf{A}^\top(\mathbf{A}\mathbf{X}-\mathbf{Y}) + 2\lambda\mathbf{X}
\end{eqnarray*}
Since $\frac{\partial J(\mathbf{\mathbf{X}})}{\partial \mathbf{X}}=0$ we have:
\begin{eqnarray*}
2\mathbf{A}^\top(\mathbf{A}\mathbf{X}-\mathbf{Y}) + 2\lambda\mathbf{X}&=&0 \\
\mathbf{A}^\top\mathbf{A}\mathbf{X} - \mathbf{A}^\top\mathbf{Y} + \lambda\mathbf{X} &=& 0\\
(\mathbf{A}^\top\mathbf{A}+\lambda\mathbb{I})\mathbf{X} &=&  \mathbf{A}^\top\mathbf{Y} \\
\mathbf{X} &=& (\mathbf{A}^\top\mathbf{A}+\lambda\mathbb{I})^{-1}  \mathbf{A}^\top\mathbf{Y}
\end{eqnarray*}

$\blacksquare$

\section{Bias and Variance}\label{app:biasandvariance}

The OLS bias can be obtained as:

\begin{eqnarray*}
Bias(\hat{f}(\hat{\mathbf{X}})) &=& E[\hat{\mathbf{X}}] - \mathbf{X} \\
&=& E[ (\mathbf{A}^\top \mathbf{A})^{-1}\mathbf{A}^\top \mathbf{Y}] - \mathbf{X} \\
&=& E[ (\mathbf{A}^\top \mathbf{A})^{-1}\mathbf{A}^\top (\mathbf{AX})] - \mathbf{X}  \\
&=& \mathbf{X}  - \mathbf{X}  \\
&=&  0
\end{eqnarray*}

The Ridge Regression bias can be obtained as:

\begin{eqnarray*}
\mathbf{X}(\lambda) &=&( \mathbf{A}^\top \mathbf{A} + \lambda \mathbb{I})^{-1}\mathbf{A}^\top \mathbf{Y} \\
&=& (\mathbb{I} + \lambda (\mathbf{A}^\top \mathbf{A})^{-1})^{-1} (\mathbf{A}^\top \mathbf{A})^{-1}\mathbf{A}^\top \mathbf{Y} \\
&=&  (\mathbb{I} + \lambda (\mathbf{A}^\top \mathbf{A})^{-1})^{-1}  \hat{\mathbf{X}} \\
&=& \mathbf{W} \hat{\mathbf{X}} 
\end{eqnarray*}

\noindent where $\mathbf{W}  = (\mathbb{I} + \lambda (\mathbf{A}^\top
\mathbf{A})^{-1})^{-1}  $ it is defined for simplicity. Ridge
regression bias is then obtained as:

\begin{eqnarray*}
Bias(\mathbf{X}(\lambda)) &=& E[\mathbf{X}(\lambda)] - \mathbf{X} \\
&=& E[\mathbf{W}\hat{\mathbf{X}}] - \mathbf{X} \\
&=&  \mathbf{W} \mathbf{X} - \mathbf{X} \neq 0 
\end{eqnarray*}


The variance of OLS is:

\begin{equation*}
Var(\hat{\mathbf{X}}) = \sigma^2 (\mathbf{A}^\top \mathbf{A} )^{-1}
\end{equation*}

\noindent and the variance of ridge regression is:

\begin{eqnarray*}
Var(\mathbf{X}(\lambda)) &=& Var(\mathbf{W}\hat{\mathbf{X}}) \\
&=& E[(\mathbf{W}\hat{\mathbf{X}}-E[\mathbf{W}\hat{\mathbf{X}}])(\mathbf{W}\hat{\mathbf{X}}-E[\mathbf{W}\hat{\mathbf{X}}])^\top] \\
&=& \mathbf{W}E[(\hat{\mathbf{X}}-E[\hat{\mathbf{X}}])(\hat{\mathbf{X}}-E[\hat{\mathbf{X}}])^\top] \mathbf{W}^\top \\
&=& \mathbf{W}Var(\hat{\mathbf{X}})\mathbf{W}^\top \\
&=& \sigma^2 \mathbf{W}(\mathbf{A}^\top \mathbf{A} )^{-1}\mathbf{W}^\top
\end{eqnarray*}

\section{Ridge regression shows an
increasing squared bias and a decreasing variance} \label{app:rrbiasvar}

\textbf{Proof}\quad

Since $Bias(\mathbf{X}(\lambda)) \neq 0$ this imply that

\begin{equation*}
Bias(\mathbf{X}(\lambda)) ^2 > 0 
\end{equation*}

\noindent we know that $\mathbf{A}^\top
\mathbf{A}$ has an eigenvalue decomposition $\mathbf{A}^\top
\mathbf{A} = \mathbf{V} \Sigma \mathbf{V}^{-1}$.


\begin{eqnarray*}
Var(\mathbf{X}(\lambda)) &=& \sigma^2 \mathbf{W}(\mathbf{A}^\top \mathbf{A} )^{-1}\mathbf{W}^\top\\
Var(\mathbf{X}(\lambda)) &=& \sigma^2 (\mathbb{I} + \lambda (\mathbf{A}^\top
\mathbf{A})^{-1})^{-1} (\mathbf{A}^\top \mathbf{A} )^{-1}((\mathbb{I} + \lambda (\mathbf{A}^\top
\mathbf{A})^{-1})^{-1} )^\top \\
&=& \sigma^2 (\mathbb{I} + \lambda (\mathbf{V} \Sigma \mathbf{V}^{-1})^{-1} (\mathbf{V} \Sigma \mathbf{V}^{-1})^{-1}((\mathbb{I} + \lambda (\mathbf{V} \Sigma \mathbf{V}^{-1})^{-1})^{-1} )^\top \\
&=& \sigma^2 \mathbf{V} (\mathbb{I} + \lambda \Sigma^{-1})^{-1} \Sigma^{-1} (\mathbb{I} + \lambda \Sigma^{-1})^{-1}  \mathbf{V}^{-1}\\
&=& \sigma^2 \mathbf{V} \mathbf{D}^{-1} \mathbf{V}^{-1}
\end{eqnarray*}

\noindent where $\mathbf{D}^{-1}  = (\mathbb{I} + \lambda \Sigma^{-1})^{-1} \Sigma^{-1} (\mathbb{I} + \lambda \Sigma^{-1})^{-1}$ is a diagonal matrix with coefficients:

\begin{equation*}
d_i = \frac{\sigma_i^{-1}}{(1+\lambda\sigma_i^{-1})^2}
\end{equation*}

$\blacksquare$

\section{Efficient computation}\label{app:effcomp}
In regression problems we always require getting a matrix inverse which is
computationally expensive. However, in stream data problems there is a way to
obtain a matrix inverse approximation using the Sherman-Morrison-Woodbury. This
formula allows to get the inverse of a matrix $\mathbf{A+uv^\top}$ if we previously calculated the
inverse of $\mathbf{A}$ as follows:

\begin{equation}
\label{eq:SMW}
(\mathbf{A+uv^\top})^{-1}=\mathbf{A}^{-1}-
\frac{\mathbf{A^{-1}uv^\top A^{-1}}}{\mathbf{1+v^\top A^{-1}u}}
\end{equation}


Alternatively, Coleman and
Sun~\cite{coleman+sun2010} presented an iterative algorithm which
uses $\mathbf{X}(\lambda)$ to approximate $\mathbf{X}(0)$.


Using the compact SVD (shown in equation~(\ref{eq:compactsvd}))
$\mathbf{X}(\lambda)$ is expressed as follows:


\begin{eqnarray}
\label{eq:optsolRRsvd}
\mathbf{X}(\lambda) & = & (\mathbf{A}^\top \mathbf{A}+ \lambda
\mathbb{I})^{-1}\mathbf{A}^\top \mathbf{Y} \nonumber \\
& = &\mathbf{V}_1(\Sigma_1^2+\lambda \mathbb{I})^{-1}\mathbf{\Sigma_1
U_1^\top Y}
\end{eqnarray}

\noindent where is easy to see that $\mathbf{X}(\lambda) \rightarrow
\mathbf{\hat{X}}=\mathbf{V}_1 \mathbf{\Sigma}_1^{-1}\mathbf{U}_1^\top
\mathbf{Y}$ as $\lambda \rightarrow 0$. 

The method consists in obtaining $\mathbf{X}(\lambda)$ and then refine
by adding more terms of its Taylor expansion to approximate
$\mathbf{X}(0) = \mathbf{\hat{X}}$. The Taylor expansion about $\lambda_0$ is:

\begin{equation}
\label{eq:taylor}
    \mathbf{X}(\lambda)=\mathbf{X}(\lambda_0) + \sum_{k=1}^\infty
    \mathbf{s}_k(\lambda-\lambda_0)^{k}
\end{equation}


\noindent where $\mathbf{s}_k=\frac{1}{k!}\mathbf{X}(\lambda)^{(k)}$
and $\mathbf{X}(\lambda)^{(k)}$ is obtained by taking differences 
$\frac{\partial}{\partial \lambda}$ of:  

\begin{eqnarray*}
(\mathbf{A}^\top \mathbf{A}+ \lambda\mathbb{I}) \mathbf{X}(\lambda) & = & \mathbf{A}^\top \mathbf{Y}\\
(\mathbf{A}^\top \mathbf{A}+ \lambda\mathbb{I}) \mathbf{X}(\lambda)^{(1)} + \mathbf{X}(\lambda)& = & 0 \\
\mathbf{X}(\lambda)^{(1)}  &=& -(\mathbf{A}^\top \mathbf{A}+ \lambda\mathbb{I}) ^{-1} \mathbf{X}(\lambda) \\
\mathbf{X}(\lambda)^{(2)}  &=& -2(\mathbf{A}^\top \mathbf{A}+ \lambda\mathbb{I}) ^{-1} \mathbf{X}(\lambda)^{(1)} \\
& \vdots & \\
\mathbf{X}(\lambda)^{(k)}  &=& -k(\mathbf{A}^\top \mathbf{A}+ \lambda\mathbb{I}) ^{-1} \mathbf{X}(\lambda)^{(k-1)} 
\end{eqnarray*}



\noindent for $\lambda=0$ we have:

\begin{equation}
\label{eq:taylor}
    \mathbf{X}(0)=\mathbf{X}(\lambda_0) + \sum_{k=1}^\infty
     (-1)^k \mathbf{s}_k \lambda_0^k
\end{equation}


\noindent therefore in order to ensure convergence, we can see that $\lambda_0$ cannot be large. 

The algorithm for computing $\mathbf{X}(0)$ is the following:

\begin{algorithm}[H]
\begin{algorithmic}[1]
\REQUIRE $\,$ \\
$\mathbf{A}$: design matrix \\
$\mathbf{Y}$: response matrix \\
$\lambda$: rank deficient parameter \\
\ENSURE  $\,$ \\
$\mathbf{X}$: parameters \\
\STATE $\mathbf{M}=\mathbf{A^\top A}$ \\
\STATE Initialize $\mathbf{Q R}=\mathbf{M}$ \\
\STATE $\mathbf{X} = \mathbf{R^{-1}Q^\top Y}$ \\
\STATE $\mathbf{s} = \mathbf{X}$ \\
\FOR { $i = 1,2,3,\dots$ }
	\STATE $\mathbf{s} =
        -(\mathbf{M}+\lambda\mathbb{I})^{-1}\mathbf{s}$\\
	\STATE $\mathbf{X}=\mathbf{X} + (-1)^i {\color{red}\mathbf{s}
        \lambda^i}$
\ENDFOR
\end{algorithmic}
\caption{Algorithm for handling rank deficient matrices}
\label{alg:coleman}
\end{algorithm}

The algorithm~\ref{alg:coleman} solves equation~(\ref{eq:taylor}). However, this
version is unstable since typically $\|\mathbf{s}\|$ is very large and
$\lambda^i$ is very small ($\lambda$ is small).

The following algorithm shows a more stable version of
algorithm~\ref{alg:coleman}.


\begin{algorithm}[H]
\begin{algorithmic}[1]
\REQUIRE $\,$ \\
$\mathbf{A}$: design matrix \\
$\mathbf{Y}$: response matrix \\
$\lambda$: rank deficient parameter \\
\ENSURE  $\,$ \\
$\mathbf{X}$: parameters \\
\STATE $\mathbf{M}=\mathbf{A^\top A}$ \\
\STATE Initialize $\mathbf{Q R}=\mathbf{M}$ \\
\STATE $\mathbf{X} = \mathbf{R^{-1}Q^\top Y}$ \\
\STATE $\mathbf{t} = \mathbf{X}$ \\
\FOR { $i = 1,2,3,\dots$ }
        \STATE $\mathbf{t} =\lambda \mathbf{t}$  
        \STATE $\mathbf{t} =  -(\mathbf{M}+\lambda\mathbb{I})^{-1}\mathbf{t}$
	\STATE $\mathbf{X}=\mathbf{X} + \mathbf{t}$
\ENDFOR
\end{algorithmic}
\caption{Algorithm for handling rank deficient matrices improved}
\label{alg:colemanimproved}
\end{algorithm}

Both algorithms are equivalent, but algorithm~\ref{alg:colemanimproved} is more
stable and converges in typically less than 10 steps.
It is important to notice that the QR factorisation is computed only once and it
is computationally less expensive than the SVD.


