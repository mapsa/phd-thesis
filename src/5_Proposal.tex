
\chapter{Online VECM proposal}
\section{\uppercase{Methodology}}
\label{sec:methodology}
\noindent 
\subsection{Online VECM} \label{sec:proposal}

Since VECM parameter estimation is computationally expensive, we propose an
online version of VECM (OVECM).  OVECM considers only a sliding window of the
most recent data. Moreover, since cointegration vectors represent long-run
relationships which vary little in time, OVECM determines firstly if they require calculation. 

OVECM also implements matrix optimizations in order to reduce execution time,
such as updating matrices with new data, removing old data and introducing new
cointegration vectors.

Algorithm~\ref{alg:proposal} shows our OVECM proposal which considers the
following:

\begin{itemize}
\item The function \texttt{getJohansen} returns cointegration vectors given by
the Johansen method considering the trace statistic test at 95\%
significance level.
\item The function \texttt{vecMatrix} returns the matrices~(\ref{eq:vecA})
and~(\ref{eq:vecB}) that allows VECM to be solved.
\item The function \texttt{vecMatrixOnline} returns the
matrices~(\ref{eq:vecA}) and~(\ref{eq:vecB}) aggregating new data and removing
the old one, avoiding calculation of the matrix $\mathbf{A}$ from scratch.
\item Out-of-sample outputs are saved in the variables 
$\mathbf{Y}_{\text{true}}$ and $\mathbf{Y}_{\text{pred}}$.
\item The model is solved using OLS.
\item In-sample outputs are saved in the variables $\Delta
\mathbf{y}_{\text{true}}$ and $\Delta \mathbf{y}_{\text{pred}}$.
\item The function \texttt{mape} gets the in-sample MAPE for the $l$ time
series.
\item Cointegration vectors are obtained at the beginning and when they are required to be updated. This updating is decided based on the in-sample MAPE of the last $n$ inputs. The average of all
MAPEs are stored in the variable $e$. If the average of MAPEs
($\texttt{mean}(e)$) is above a certain error given by the mean\_error threshold, cointegration vectors are updated.
\item If new cointegration vectors are required, the function
\texttt{vecMatrixUpdate} only updates the corresponding columns of matrix
$\mathbf{A}$ affected by those vectors (see equation~\ref{eq:vecA}).
\end{itemize}

\begin{algorithm}[ht]
\begin{algorithmic}[1]
\REQUIRE $\,$ \\
$\mathbf{y}$: matrix with $N$ input vectors and $l$ time series\\
$p$: number of past values \\
$L$: sliding window size ($L<N$) \\
$\text{mean\_error}$: MAPE threshold \\
$n$: interpolation points to obtain MAPE \\
\ENSURE  $\,$ \\
$\{\mathbf{y}_{\text{pred}}[L+1],\dots,\mathbf{y}_{\text{pred}}[N]\}$: model predictions 
\FOR { $i =0$ to $N-L$ }
    \STATE $\mathbf{y}_i \gets \mathbf{y}[i:i+L]$
	\IF {i = 0 }
	    \STATE{$v \gets \texttt{getJohansen}(\mathbf{y}_i,p)$}
	    \STATE{$[\mathbf{A} \quad \mathbf{B}] \gets
        \texttt{vecMatrix}(\mathbf{y}_i,p,v)$}
	\ELSE
	    \STATE{$[\mathbf{A} \quad \mathbf{B}] \gets
        \texttt{vecMatrixOnline}(\mathbf{y}_i,p,v,\mathbf{A},\mathbf{B})$}
        %\STATE $\Delta \mathbf{Y}_{\text{true}}[i] \gets \mathbf{B}[-1,:]$
        \STATE $\Delta \mathbf{Y}_{\text{pred}}[i] \gets \mathbf{A}[-1,:] \times \mathbf{X}$
    \ENDIF
    \STATE $\mathbf{X} \gets \texttt{OLS} (\mathbf{A},\mathbf{B})$%\mathbf{(A^\top A)^{-1}A^\top B}$
    %\STATE $\Delta \mathbf{y}_{\text{true}} \gets \mathbf{B}[-n,:]$
    %\STATE $\Delta \mathbf{y}_{\text{pred}} \gets \mathbf{A}[-n,:] \times \mathbf{X}$
    \STATE $e \gets \texttt{mape}(\mathbf{B}[-n,:],\mathbf{A}[-n,:] \times \mathbf{X})$
    %\STATE $e \gets \texttt{mape}(\Delta \mathbf{y}_{\text{true}}, \Delta
    %\mathbf{y}_{\text{pred}})$
    \IF {$\texttt{mean}(e) > \text{mean\_error}$}
	    \STATE{$v \gets \texttt{getJohansen}(\mathbf{y}_i,p)$}
	    \STATE{$\mathbf{A} \gets
        \texttt{vecMatrixUpdate}(\mathbf{y}_i,p,v,\mathbf{A})$}
        \STATE $\mathbf{X} \gets \texttt{OLS} (\mathbf{A},\mathbf{B})$%\mathbf{(A^\top A)^{-1}A^\top B}$
        %\STATE $\mathbf{X} \gets \mathbf{(A^\top A)^{-1}A^\top B}$
    \ENDIF
\ENDFOR
\STATE $\mathbf{Y}_{\text{true}} \gets \mathbf{Y}[L+1:N]$
\STATE $\mathbf{Y}_{\text{pred}}\gets 
\mathbf{Y}[L:N-1] +\Delta \mathbf{Y}_{\text{pred}}$
%\STATE $\text{MAPE} \gets \texttt{mape}(\Delta \mathbf{Y}_{\text{true}}, \Delta
%\mathbf{Y}_{\text{pred}})$
\end{algorithmic}
\caption{OVECM: Online VECM}
\label{alg:proposal}
\end{algorithm}



Our proposal was compared against VECM and ARIMA. Both algorithms were adapted
to an online context in order to get a reasonable comparison with our proposal
(see algorithms \ref{alg:SLVECM} and \ref{alg:SLARIMA}). VECM and ARIMA are
called with a sliding window of the most recent data, whereby the models are
updated at every time step.

\begin{algorithm}[ht]
\begin{algorithmic}[1]
\REQUIRE $\,$ \\
$\mathbf{y}$: matrix with $N$ input vectors and $l$ time series\\
$p$: number of past values \\
$L$: sliding window size ($L<N$) \\
\ENSURE  $\,$ \\
$\{ \mathbf{y}_{\text{pred}}[L+1],\dots,\mathbf{y}_{\text{pred}}[N]\}$: model predictions 
\FOR { $i =0$ to $N-L$ }
    \STATE $\mathbf{y}_i \gets \mathbf{y}[i:i+L+1]$
        \STATE $model = VECM(\mathbf{y}_i, p)$
        \STATE $\mathbf{Y}_{\text{pred}}[i] = model.predict(\mathbf{y}[i+L])$
\ENDFOR
\STATE $\mathbf{Y}_{\text{true}} = \mathbf{y}[i+L+1:N] $
\end{algorithmic}
\caption{SLVECM: Sliding window VECM}
\label{alg:SLVECM}
\end{algorithm}

Since we know out time series are I(1) SLARIMA is called with $d=1$. ARIMA is
executed for every time series. 

\begin{algorithm}[ht]
\begin{algorithmic}[1]
\REQUIRE $\,$ \\
$\mathbf{y}$: matrix with $N$ input vectors and $l$ time series\\
$p$: autoregressive order \\
$d$: integrated order\\
$q$: moving average order\\
$L$: sliding window size ($L<N$) \\
\ENSURE  $\,$ \\
$\{ \mathbf{y}_{\text{pred}}[L+1],\dots,\mathbf{y}_{\text{pred}}[N]\}$: model predictions 
\FOR { $i =0$ to $N-L$ }
\FOR { $j =0$ to $l-1$ }
    \STATE $\mathbf{y}_i \gets \mathbf{y}[i:i+L+1,j]$
        \STATE $model = ARIMA(\mathbf{y}_i, (p,d,q))$
        \STATE $\mathbf{Y}_{\text{pred}}[i,j] = model.predict(\mathbf{y}[i+L,j])$
\ENDFOR
\ENDFOR
\STATE $\mathbf{Y}_{\text{true}} = \mathbf{y}[i+L+1:N,:] $
\end{algorithmic}
\caption{SLARIMA: Sliding window ARIMA}
\label{alg:SLARIMA}
\end{algorithm}

Both OVECM and SLVECM time complexity is dominated by Johansen method which is
$O(n^3)$. Thus, both algorithms order is $O(Cn^3)$ where $C$ is the number of
iterations. 

\subsection{Evaluation methods} \label{sec:evaluation}

Forecast performance is evaluated using different methods. We have chosen three
measures usually used:
\begin{description}
\item[MAPE] Mean Average Percent Error which presents forecast errors as a
percentage.
\begin{equation}\label{eq:MAPE}
\text{MAPE} = \frac{1}{N} \sum_{t=1}^{N} 
\frac{\left|\mathbf{y}_t-\hat{\mathbf{y}}_t\right|}{\left|\mathbf{y}_t\right|}
 \times 100 
\end{equation}
\item[MAE] Mean Average Error which measures the distance between forecasts to the
true value.
\begin{equation}\label{eq:MAE}
\text{MAE} = \frac{1}{N} \sum_{t=1}^{N} 
\left| 
\mathbf{y}_t-\hat{\mathbf{y}}_t
\right| 
\end{equation}
\item[RMSE] Root Mean Square Error also measures the distance between forecasts
to the true values but, unlike MAE, large deviations from the true value have a
large impact on RMSE due to squaring forecast error.
\begin{equation}\label{eq:RMSE}
\text{RMSE} = \sqrt{
\frac{\displaystyle \sum_{t=1}^{N} (\mathbf{y}_t-\hat{\mathbf{y}}_t)^2}{N}}
\end{equation}
\end{description}


\subsection{Model selection} \label{sec:pselection}
Akaike Information Criterion (AIC) is often used in model selection where AIC
with smaller values are preferred since they represent a trade-off between bias
and variance.  AIC is obtained as follows:

\begin{equation}
\label{eq:aicformula}
AIC = \underset{\text{bias}}{-\frac{2l}{N}} + 
\underset{\text{variance}}{\frac{2k}{N}}
\end{equation}

\noindent where 

\begin{description}
\item[l] is the loglikelihood function
\item[k] number of estimated parameters
\item[N] number of observations
\end{description}

Loglikelihood function is obtained from the Residual Sum of Squares (RSS):

\begin{equation}
\label{eq:ll}
l = -\frac{N}{2} \left(1 + ln(2\pi) + ln\left(\frac{RSS}{N}\right)\right) 
\end{equation}

    \begin{tikzpicture}[node distance=2cm]
    %\node (start) [startstop] {Start};
    %\node (in1) [io, below of=start] {Input};
    \node (in1) [io] {Input: y, p, L, $\epsilon$, n};
    \node (pro1) [process, below of=in1] {$y_i = y[i:i+L]$};
    \node (dec1) [decision, below of=pro1, yshift=-0.5cm] {i = 0};
    \node (pro2a) [process, below of=dec1, yshift=-0.5cm] {$[\mathbf{A} \quad \mathbf{B}] =\texttt{ovecm}(\mathbf{y}_i,p,v,\mathbf{A},\mathbf{B})$
                                                        $\Delta \mathbf{Y}_{\text{true}}[i] = \mathbf{B}[-1,:]$
                                                        $\Delta \mathbf{Y}_{\text{pred}}[i] = \mathbf{A}[-1,:] \times \mathbf{X}$};

    \node (pro2b) [process, right of=dec1, xshift=4cm] {$v = \texttt{Johansen}(\mathbf{y}_i,p)$ $[\mathbf{A} \quad 
                                                                \mathbf{B}] = \texttt{vecm}(\mathbf{y}_i,p,v)$};
    \node (pro3) [process, below of=pro2a, yshift=-0.5cm] {$\mathbf{X} = \text{OLS} (\mathbf{A},\mathbf{B})$ \\
                                                        $\mathbf{y}_{\text{true}}[i] = \mathbf{y}_i[-n,:]$ \\
                                                        $\mathbf{y}_{\text{pred}}[i] = \mathbf{A}[-n,:] \times \mathbf{X} +$ \\$ \mathbf{y}_i[-n-1,:-1]$ \\
                                                        $e = \texttt{mape}(\mathbf{y}_{\text{true}}, \mathbf{y}_{\text{pred}})$};

    \node (dec2) [decision, below of=pro3, yshift=-0.5cm] { $e > \epsilon $ };

    \node (pro4a) [process, right of=dec2, xshift=4cm] {$v = \texttt{Johansen}(\mathbf{y}_i,p)$ \\
                                                        $\mathbf{A} = \texttt{uvecm}(\mathbf{y}_i,p,v,\mathbf{A})$ \\
                                                        $\mathbf{X} = \texttt{OLS} (\mathbf{A},\mathbf{B})$};

   
    \node (pro5) [process, left of=pro2a, xshift=-4cm] {$ i++ $}; 
   
    \node (dec3) [decision, below of=dec2, yshift=-0.5cm] { $i > it $ };
    
    \node (out1) [io, below of=dec3] {Output: $\Delta \mathbf{y}_{true}, \Delta \mathbf{y}_{pred}$};
    
    %\node (stop) [startstop, below of=out1] {Stop};
    
    %\draw [arrow] (start) -- (in1);
    \draw [arrow] (in1) -- (pro1);
    \draw [arrow] (pro1) -- (dec1);
    \draw [arrow] (dec1) -- node[anchor=east] {no} (pro2a);
    \draw [arrow] (dec1) -- node[anchor=south] {si} (pro2b);
    \draw [arrow] (pro2a) -- (pro3);
    \draw [arrow] (pro2b) |- (pro3);
    \draw [arrow] (pro3) -- (dec2);
    \draw [arrow] (dec2) --  node[anchor=south] {si} (pro4a);
    \draw [arrow] (pro4a) |- (dec3);
    
    \draw [arrow] (dec2) --  node[anchor=east] {no} (dec3);
    \draw [arrow] (dec3.west) -| node[anchor=east] {no} (pro5);
    \draw [arrow] (pro5) |- (pro1);
    \draw [arrow] (dec3) --  node[anchor=east] {si} (out1);
    %\draw [arrow] (out1) -- (stop);
    \end{tikzpicture}
