%% 20170814_Paola_Chapter_3_again.tex

\red{\section{\Large AUGUST 14, 2017 -- Chapter 3 once again!}}


\begin{description}[style=unboxed,leftmargin=0cm,itemsep=3ex]

\myitem{P 36, section 3.2, L 6}
It says:\quad ``...in finding a learning function $f:X\to Y$
which models... etc.''
It is not said here what are $X$ and $Y$; 
further on it is said. 
This is not usual in a scientific text.
Thus, I would suggest a text like this:  

\red{``...in finding a learning function $f:X\to Y$, where $X\subseteq\R^m$
and $Y\subset\R$, which models a set of examples $S\subseteq X\times Y$,
called {\em training sets}, which have been drawn randomly and
independently from $X\times Y$ according to \blue{an} unknown joint
distribution function $p(x,y)$ on $X\times Y$:
$$
S=\big\{ (\mathbf{x}_k,y_k)\in X\times Y \mid k=1,\dots,n \big\}
$$
Here, $Y=\R$ in case of regression problems and $Y=\{-1,1\}$ if it is
a classification problem.''}

\blue{However, below there is a deeper discussion of these facts and
I propose you a more comprehensive composition.}


%% \myitem{P 36, section 3.2, L -4}
%% The expression $S\in\mathscr{T}=\mathscr{P}(X\times Y)$ seems
%% unnecessarily complicated.

%% Isn't it the same (and much simpler) to say that the training sets
%% fulfill $S\subseteq X\times Y$ ?

%% The introduction of $\mathscr{T}$ suggests that you want to have the
%% possibility of choosing \red{several} training sets $S$ taken from
%% $\mathscr{T}=\mathscr{P}(X\times Y)$, but this is unnecessary since
%% if you indeed had several training sets, say, $S_\alpha,\;S_\beta,\;\dots$, 
%% you coud always consider a super-training set
%% $S=S_\alpha \cup S_\beta \cup \dots$ and you would achieve the same generality.

%% Thus in the sequel I shall abandone the set $\mathscr{T}$ and shall assume
%% that we have just \red{ONE} training set $S$ and that this is a \red{subset}
%% of $X\times Y$: $S\subseteq X\times Y$.

\myitem{P 36, section 3.2, L -3}
Suddenly the symbol \red{$\mathscr{H}$} appears on the stage!
But it is not defined before or in the same line!
Only after the formula for $\mathcal{A}$ there is a hint about
$\mathscr{H}$.
This writing style is highly non-standard in math!

\myitem{P 36-37, section 3.2. Some remarks trying to make sense of all these!}

\mbox{}\newline
After reading your definitions several times, it seems to me that:

\begin{enumerate}

\item
Which are the references for the many definitions in these pages ?

\item
The most basic data sets you have are $X\subseteq\R^m$ and $Y\subseteq\R$.
You distinguish two cases: $Y=\R$, the case of regression problems;
and $Y=\{-1,1\}$, the case of classification problems.

\item
Then you have a training set $S\subseteq X\times Y$ of the form:
$$
S=\big\{ (\mathbf{x}_k,y_k)\in X\times Y \mid k=1,\dots,n \big\}\,,
$$
where the members of $S$ have been drawn randomly and independently
according to a unknown joint distribution function
$p(\mathbf{x},y)$ on $X\times Y$.

\item
The problem is to find a function $f:X\to Y$ such that:
\begin{itemize}
\item
it \red{models} the training set $S$ in the sense that
$f(\mathbf{x}_k)=y_k$ for all $k=1,\dots,n$; \\
\red{Is this true? If not, then in what sense $f$ models $S$ ?}

\item
it has a \red{predictive capacity} in the sense that $f(\mathbf{x})=y$
for a newly arrived data $(\mathbf{x},y)\in X\times Y\setminus S$.
\end{itemize}
These conditions seem to be too restrictive.
We can weaken them a little by accepting that the function $f:X\to Y$ should:
\begin{itemize}
\item
just \red{closely models} the training set $S$ in the sense that
$f(\mathbf{x}_k)\approx y_k$ with an error $|f(x_k)-y_k|<\varepsilon$ for
a given $\varepsilon>0$ and all $k=1,\dots,n$; \\
\red{Is this true? If not, then in what sense $f$ models $S$ ?}
\item
have an \red{approximate predictive capacity} in the sense that
$f(\mathbf{x})\approx y$ with error $|f(\mathbf{x})-y|<\varepsilon$
for a newly arrived data $(\mathbf{x},y)\in X\times Y\setminus S$.
\end{itemize}
\item 
This function $f:X\to Y$ should be searched after (and hopefully found)
among a set of functions $\mathscr{H}$ from $X$ to $Y$.

The set of functions $\mathscr{H}$ is called \red{hypothesis space}.

Technically speaking, $\mathscr{H}$ is a subset of the set $Y^X$
of \red{all} functions $\varphi:X\to Y$, but this is mathematically too
technical for our purposes in this work.

For the search process it would be convenient to introduce a topology
on $\mathscr{H}$, preferrable a norm topology (like an $L^p$-norm for
instance) and to require topological completeness.
It would be even better if we would require that $\mathscr{H}$ is Banach
space, but this would be perhaps too technical for our goals here.


\item
There is also a learning algorithm $\mathcal{A}$ which allows $f$
to be found in the \red{hypothesis space} $\mathscr{H}$.

We discussed this subject in October last year.
Please see my remarks from that month.

You now interpret $\mathcal{A}$ as an application (function) from
$\mathscr{T}\subseteq\mathscr{P}(X\times Y)$ to $\mathscr{H}$, 
$\mathcal{A}:S\mapsto\mathcal{A}(S)=f$.

\red{Note that you write $\mathscr{T}=\mathscr{P}(X\times Y)$, but
$\mathscr{T}\subseteq\mathscr{P}(X\times Y)$ gives you more freedom.}

I think there is still a conceptual ``darkness'' associated with 
``$S\in\mathscr{T}\subseteq\mathscr{P}(X\times Y)$''.

The problem is \red{to find a learning function $f$}, with the
properties stated above, for some particular (financial) experiment,
but it is one and only one well defined experiment, for instance,
the analysis of a vector valued (financial) time series (this would
be the $\mathbf{x}_k=\left[x_k^1,\dots,x_k^m\right]^T$ for the
(time-) instants $k=1,\dots,n$, but what would represent the 
``$y_k\in\R$''?
But never mind, it will be a real number (possibly $\pm1$) for each $k$.

Now comes the delicate point: for this one and only one experiment
you have:
\begin{itemize}
\item[1.]
\red{Either} one and only one data set 
$S\in\mathscr{T}\subseteq\mathscr{P}(X\times Y)$
and hence your algorithm $\mathcal{A}$ acting on $S$
produces one and only one learning function $f$, $\mathcal{A}(S)=f$,
for the whole experiment.

In this case it doesn't make sense (you could, but it doesn't help too
much!) to consider the algorithm $\mathcal{A}$
as a mapping $\mathcal{A}:\mathscr{T}\to\mathscr{H}$ since $\mathscr{T}$
has only one member, namely, $\mathscr{T}=\big\{S\big\}$.
\item[2.]
\red{Or} several different data sets
$S_\alpha,S_\beta,\dots\in\mathscr{T}\subseteq\mathscr{P}(X\times Y)$
of the same one and only one experiment, and hence your algorithm
$\mathcal{A}$ acting on $S_\alpha$, $S_\beta$, $\dots$ should produce
learning functions $\mathcal{A}(S_\alpha)=f_\alpha$, 
$\mathcal{A}(S_\beta)=f_\beta$, $\dots$

In this case you could have 
$\mathscr{T}=\big\{ S_\alpha,S_\beta,\dots\big\}$
and it make sense to consider the algorithm $\mathcal{A}$ as a mapping
$\mathcal{A}:\mathscr{T}\to\mathscr{H}$:
$$
S_\alpha\mapsto \mathcal{A}(S_\alpha)=f_\alpha\in\mathscr{H}\,,\quad
S_\beta \mapsto \mathcal{A}(S_\beta )=f_\beta \in\mathscr{H}\,,\quad\dots
$$
\end{itemize}


\item
\blue{\bf An example of Case 2. as above, to fix the ideas!}
This would be the case if you consider the data sets 
$S_\alpha,S_\beta,\dots\in\mathscr{T}\subseteq\mathscr{P}(X\times Y)$
as \red{segments} or \red{windows} of a time series
$\big\{(\mathbf{x}_t,y_t)\big\}_{t\in\Z}$ of vectors 
$(\mathbf{x}_t,y_t)\in\R^m\times\R$, say
\begin{align*}
S_\alpha &= \bigg\{ 
\left( \mathbf{x}_{t_\alpha}, y_{t_\alpha} \right), 
\left( \mathbf{x}_{t_\alpha+1}, y_{t_\alpha+1} \right), \dots, 
\left( \mathbf{x}_{t_\alpha+n-1}, y_{t_\alpha+n-1} \right) \bigg\}\,, \\
S_\beta &= \bigg\{ 
\left( \mathbf{x}_{t_\beta}, y_{t_\beta} \right), 
\left( \mathbf{x}_{t_\beta+1}, y_{t_\beta+1} \right),\dots, 
\left( \mathbf{x}_{t_\beta+n-1}, y_{t_\beta+n-1} \right) \bigg\}\,, \\
\vdots & \rule{10em}{0ex}\vdots
\end{align*}
for $t_\alpha,t_\beta,\dots\in\Z$, which without restriction we can assume
to be increasing: $t_\alpha<t_\beta<\dots$.

Note that here all data sets $S_\alpha,S_\beta,\dots$ have the same cardinality
(windows of the same length), namely $n$.
But this is not obligatory: windows could have different length, but
computations would then be more cumbersome.

Also note that nothing is said about the convergence of the sequence
$f_\alpha,f_\beta,\dots$
In fact this sequence may very well not converge in $\mathscr{H}$.
The terms of this sequence are functions that model the behaviour of the
time series within the particular window associated to each data set
$S_\alpha,S_\beta,\dots$ and what is expected is that the corresponding
learning functions $f_\alpha,f_\beta,\dots$ have good predictive capacity,
namely that: 
\begin{align*}
\widehat{y}_{t_\beta}
&:= \mathcal{A}(S_\alpha)(\mathbf{x}_{t_\beta})
\equiv f_\alpha(\mathbf{x}_{t_\beta}) \approx y_{t_\beta} \quad
\text{with an error $\big|\widehat{y}_{t_\beta}-y_{t_\beta}\big|<\varepsilon$}\,, \\
\widehat{y}_{t_\gamma}
&:= \mathcal{A}(S_\beta)(\mathbf{x}_{t_\gamma})
\equiv f_\beta(\mathbf{x}_{t_\gamma}) \approx y_{t_\gamma} \quad
\text{with an error $\big|\widehat{y}_{t_\gamma}-y_{t_\gamma}\big|<\varepsilon$}\,, \\
\vdots & \rule{10em}{0ex}\vdots
\end{align*}
Note that in these equations $f_\alpha$ is evaluated in $\mathbf{x}_{t_\beta}=$
the $X$-coordinate of the first member of the next ``window'' $S_\beta$.

Similarly, $f_\beta$ is evaluated in $\mathbf{x}_{t_\gamma}=$ the $X$-coordinate
of the first member of the next ``window'' $S_\gamma$.
Etc.

Note, however, that each data set in Case 2. is just another version of
the one and only one data set in Case 1.
Thus, Case 2. is \red{not} a generalisation of Case 1.
In fact, Case 2. can be reduced to Case 1. just by taking the union of
all data sets of Case 2.

\red{\bf Thus, it seems to me that you are considering Case 2.
I will assume this in the sequel!}

\item\label{referenceFunction}
Note that there is always an {\em observed\/} or {\em reference function\/}
$$
\mathscr{Y}:X\to Y\,,\qquad
\mathscr{Y}: \mathbf{x}_k \mapsto \mathscr{Y}(\mathbf{x}_k):=y_k\,,
\quad k=1,\dots,n\,.
$$
We will always assume that $\mathscr{Y}\in\mathscr{H}$.
In fact, $\mathscr{Y}$ is from the very beginning the most natural member
of $\mathscr{H}$!

Besides this reference function $\mathscr{Y}$ there are the learning
functions
$f\in\mathscr{H}$:
$$
f:X\to Y\,,\qquad
f: \mathbf{x}_k \mapsto f(\mathbf{x}_k)=:\widehat{y}_k\,,
\quad k=1,\dots,n\,.
$$
The {\em error function\/} associated to any $f\in\mathscr{H}$
is defined by:
$$
\texttt{Error}_f(\mathbf{x}_k)
:=|f(\mathbf{x_k})-\mathscr{Y}(\mathbf{x}_k)|=|\widehat{y}_k-y_k|\,,
\quad k=1,\dots,n\,,
$$
where the error function \texttt{Error} should ideally be small, say, 
$\texttt{Error}(\mathbf{x}_k)<\varepsilon$ for all $k=1,\dots,n$, for
a given $\varepsilon>0$.

Note that $f,\mathscr{Y}:X\to Y$ but $\texttt{Error}_f:X\to\R_0^+$.

In function-analytic notation we observe that, for $f\in\mathscr{H}$,
$\texttt{Error}_f$ is a function on $X$:
$$
\texttt{Error}_f:X\to\R_0^+\,,\qquad
\texttt{Error}_f(\mathbf{x}) 
= |f(\mathbf{x})-\mathscr{Y}(\mathbf{x})| 
= |\widehat{y} - y| \quad\forall x\in X\,.
$$
and since the functions $f$, $\mathscr{Y}$, and hence $\texttt{Error}_f$,
are defined on the whole of $X$ we can also write for any
$f\in\mathscr{H}$:
$$
\texttt{Error}_f(\mathbf{x})
= |f(\mathbf{x})-\mathscr{Y}(\mathbf{x})| 
= |f(\mathbf{x}) - y| = |\widehat{y} - y|\,,\quad \mathbf{x}\in X,\ 
\text{provided that}\ (\mathbf{x},y)\in S\in\mathscr{T}.
$$

\item\label{lossFunction}
For a given $S\in\mathscr{T}$, the algorithm $\mathcal{A}$ selects
one (and only one) learning function $f\in\mathscr{H}$ by minimising
the expected error of a {\em loss function\/}:
$$
V:Y\times Y\to\R_0^+\,,\qquad V: (y_1,y_2)\mapsto V(y_1,y_2)\,.
$$
Note that both arguments of $V$ are points in $Y$.
The prototype of this loss function in the case $Y=\R$ is the {\em metric\/}
\begin{equation}\label{BasicLossFunction}
V(y_1,y_2):=|y_1-y_2|\qquad \forall y_1,y_2\in Y=\R\,. 
\end{equation} 
Two simple loss functions for regression problemas derived from
(\ref{BasicLossFunction}) for $f\in\mathscr{H}$ are:
$$
V\big( f(\mathbf{x}),y \big) 
= |f(\mathbf{x})-y| = |\widehat{y}-y|\,,\quad 
\mathbf{x}\in X\ \text{and}\ y\in Y\ 
\text{with}\ (\mathbf{x},y)\in S\in\mathscr{T}\,,
$$
which is the so called $L^1$-{\em loss function\/} and:
$$
V\big( f(\mathbf{x}),y \big) 
= \big(f(\mathbf{x})-y\big)^2 = \big( \widehat{y}-y\big)^2\,,\quad 
\mathbf{x}\in X\ \text{and}\ y\in Y\ 
\text{with}\ (\mathbf{x},y)\in S\in\mathscr{T}\,,
$$
which is the so called {\em square\/} or $L^2$-{\em loss function\/}.

\item
Note that $V\big( f(\mathbf{x}),y \big)$, $\mathbf{x}\in X$ and $y\in Y$ 
with $(\mathbf{x},y)\in S\in\mathscr{T}$, is actually a {\em function\/}
on $X\times Y$:
$$
V\big(f(\mathbf{\cdot}),\cdot):X\times Y\to\R_0^+\,,\quad
\big( \mathbf{x},y \big) \mapsto V\big( f(\mathbf{x}),y \big)\,,\ 
\text{where}\ (\mathbf{x},y)\in S\in\mathscr{T}\,.
$$
Condition ``$(\mathbf{x},y)\in S\in\mathscr{T}$'' is a small nuisance here,
since the function ``$V\big(f(\mathbf{\cdot}),\cdot)$''  \red{is not defined
{\em for all\/} $\mathbf{x}\in X$ and {\em all\/} $y\in Y$.}
But this is inevitable, since the members of the data sets which we are
dealing with are of the form $(\mathbf{x},y)$.
It could be even the case that the variable $\mathbf{x}$ does not run on
the whole of $X$, and this is most surely the case for $y$ regarding $Y$.
When we come to compute the expected value of the function
$V\big(f(\mathbf{\cdot}),\cdot)$, we will see an easy solution to
this ``beauty defect'' by means of a clever probability distribution on
$X\times Y$ (see (\ref{expectedError}) below ).

There is another function closely related to $V\big(f(\mathbf{\cdot}),\cdot)$
but different fro that one:
$$
V\big(f(\cdot), \mathscr{Y}(\cdot)\big):X\to Y\,,\qquad
\mathbf{x}\mapsto V\big(f(\mathbf{x}), \mathscr{Y}(\mathbf{x})\big)
= \big( \widehat{y}, y \big)\,,\qquad f\in\mathscr{H}\,.
$$
Note that here we do not need the ``ugly'' condition
``$(\mathbf{x},y)\in S\in\mathscr{T}$'' because this is built-in in the
definition of the reference function $\mathscr{Y}:X\to Y$
(see (\ref{referenceFunction}) above).

\item\label{expectedError}
Consider a {\em learning function\/} $f\in\mathscr{H}$,
$f(\mathbf{x})=\widehat{y}$.
Also consider the {\em reference function\/} $\mathscr{Y}:X\to Y$, 
$\mathscr{Y}(\mathbf{x})=y$, where $(\mathbf{x},y)\in S\in\mathscr{T}$,
$S\subseteq X\times Y$.

Note that $S$ may be considered as the {\em graph\/} of $\mathscr{Y}$:
$$
\text{Graph}(\mathscr{Y})
:= \big\{ (\mathbf{x}, y)\in X\times Y \mid \mathscr{Y}(\mathbf{x})=y\big\}
= S\subseteq X\times Y\quad !
$$
There are clearly (many) probability distributions $p$ on $X\times Y$
such that:
\begin{equation}\label{probdist}
p\ \text{assigns probability 0 to all ``events''}\  
(\mathbf{x},y)\in X\times Y\setminus\text{Graph}(\mathscr{Y})
= X\times Y\setminus S\,.
\end{equation}
Consider, in addition, a given {\em loss function\/} $V:Y\times Y\to\R_0^+$.
(see (\ref{lossFunction}) above) and a probability distribution $p$ over
$X\times Y$ with property (\ref{probdist}).

The {\em expected error\/} of $V(f(\cdot),\cdot)$ may then be defined as:
$$
E\big[ V( f(\cdot),\cdot) \big]
= \int_{X\times Y} V\big( f(\mathbf{x},y \big)\,dp(\mathbf{x},y)\,.
$$
\color{red}{
Compare this last formula with your formula (3.3).
Note that in your formula it doesn't make sense to have the variables
$\mathbf{x}$ and $y$ on the left-hand-side since this variebles are
used in the integration on the righ-hand-side, hence they are 
``destroyed'' by the integration and they are not available any more:
the whole formul IS NOT a function of $\mathbf{x}$ and $y$.

Also, in these formulas adopt the convention of writing the variable
$\mathbf{x}$ in {\bf boldface} to recall that this variable is a
vector in $X=\R^m$.

}

%%%%%%% DRAFT PROPOSAL %%%%%%%%%%%%%%%%%%%%%%%%%%%%%%%%%%%%%%%%%%%%%%%%

\vspace{4ex}
\item
{\color{blue}
{\Large\bf A draft proposal.} \\[1ex]
The basic setting for supervised learning is a {\em data set\/} $X$, 
and an  {\em outcome set\/} $Y$.
For our purposes $X$ will be $\R^m$, $m\in\N$, and $Y$ either $\R$
in the case of regression problems, or $\{-1,1\}$ in the case of
classification problems.
In addition we have a plurality of {\em training sets\/}
$$
S=\bigg\{ (\mathbf{x}_k,y_k)\in X\times Y \mid k=1,\dots,n \bigg\}
\subseteq X\times Y\,,
$$
where the members of $S$ have been drawn randomly and independently
from $X\times Y$ according to an {\em unknown\/} joint distribution
function $p(\mathbf{x},y)$ on $X\times Y$.
We gather all these training sets in a subset $\mathscr{T}$ of
$\mathscr{P}(X\times Y)$.

{\em Supervised learning\/} consists then in finding a
{\em learning function\/} $f:X\to Y$ such that:

%% \begin{itemize}
%% \item
%% it \red{models} the training set $S$ in the sense that $f(x_k)=y_k$
%% for all $k=1,\dots,n$ and
%% \item
%% it has a \red{predictive capacity} in the sense that $f(x)=y$ for a
%% newly arrived data $(x,y)\not\in S$.
%% \end{itemize}
%% These conditions seem to be too restrictive.
%% We can weaken them a little by accepting that the function $f:X\to Y$ should:

\begin{itemize}
\item
it closely models the training set $S$ in the sense that
$f(\mathbf{x}_k)\approx y_k$ with an error
$|f(\mathbf{x}_k)-y_k|<\varepsilon$ for a given $\varepsilon>0$
and all $k=1,\dots,n$;
\red{Is this true? If not, then in what sense $f$ models $S$ ?}
\item
it has an approximate predictive capacity in the sense that
$f(\mathbf{x})\approx y$ with error $|f(\mathbf{x})-y|<\varepsilon$
for a newly arrived data $(\mathbf{x},y)\in X\times Y\setminus S$.
\end{itemize}
The learning function $f:X\to Y$ is to be searched after among a set
of functions $\mathscr{H}$ from $X$ to $Y$.
The set of functions $\mathscr{H}$ is called {\em hypothesis space\/}.

%% For the search process it is convenient to consider a vector space
%% structure and a topology on the hypothesis space $\mathscr{H}$.
%% A popular topology in the field is the $L^p$-norm topology for some
%% $p\in\R$ with $1\leq p\leq \infty$.
%% Topological completeness of $\mathscr{H}$ is convenient as well.
%% A convenient comprehensive requirement would be that $\mathscr{H}$ be
%% a Banach space.

In addition, there is a {\em learning algorithm\/} $\mathcal{A}$ that
associates to every data set one and only one learning function
$f\in\mathscr{H}$.
Thus $\mathcal{A}$ can be considered as a function:
$$
\mathcal{A}:\mathscr{T}\to\mathscr{H}\,,\qquad
S\mapsto\mathcal{A}(S)=f\quad\forall S\in\mathscr{T}\,.
$$ 

PLEASE CONTINUE HERE WITH THE PERTINENT REMARKS WRITTEN ABOVE !

}

\end{enumerate}

\end{description}

%% END OF FILE

