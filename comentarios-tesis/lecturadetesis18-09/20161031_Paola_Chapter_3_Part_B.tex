% 20161009_Paola_Chapter_3_Part_B.tex --- 31.10.2016

\section{Chapter 3: Remarks from 20161031}

\begin{description}[style=unboxed,leftmargin=0cm,itemsep=3ex]

\myitem{P 39, LL 11-13, \red{3.2.4 Ridge Regression. 
        Discussion of the equation for $J(\mathbf{X})$}}
I discuss here the equation for $J(\mathbf{X})$ starting from equation (3.6).
For any matrix $\mathbf{A}$ let 
$\mathbf{A}_i$ be the $i$-th row,
$\mathbf{A}^j$ be the $j$-th column, and
$\mathbf{A}_i^j$ be the $(i,j)$-element of matrix the $\mathbf{A}$.
Then:
\begin{alignat*}{2}
&\text{$(i,j)$-element of the \red{left} hand side of (3.6)}
= \mathbf{A}_i\;\mathbf{X}^j\,, &\quad& i=1:n\,,\ j=1:l\,, \\
&\text{$(i,j)$-element of the \red{right} hand side of (3.6)}
= \mathbf{Y}_i^j\,, &\quad& i=1:n\,,\ j=1:l\,.
\end{alignat*}
If $\|\cdot\|_2$ denotes the Frobenius matrix norm, we would have:
\begin{equation}\label{20161031:1}
J(\mathbf{X})
= \|\mathbf{A}\,\mathbf{X}-\mathbf{Y}\|_2^2
= \sum_{j=1}^l \left\| \mathbf{A}\,\mathbf{X}^j - \mathbf{Y}^j \right\|_2^2
= \sum_{i=1}^m \sum_{j=1}^l 
  \left( \mathbf{A}_i\,\mathbf{X}^j - \mathbf{Y}_i^j \right)^2
\end{equation}
Now the first problem is the meaning that you assign to the equality
\begin{equation}\label{20161031:2}
J(\mathbf{X})=\sum_{t=1}^N \left(f(\mathbf{x}_t)-y_t\right)^2\,.
\end{equation}
Everything is undefined here: $t,\,N,\,f,\,\mathbf{x}_t,y_t$.
Thus, the equality does not make sense.
We have to find a plausible interpretation.
It might be that the term with the sum in the right hand side of
(\ref{20161031:2}) concides with the last term in (\ref{20161031:1}).
In this interpretation, 
\begin{equation}\label{20161031:3}
f(\mathbf{x}_t) \quad\text{would be}\quad 
f_i(\mathbf{X}^j):=\mathbf{A}_i\,\mathbf{X}^j\,,
\end{equation}
but you would still need two subindices: $i$ and $j$.

One could, of course, introduce a numbering $t=j+(i-1)\,l$ of the
coefficients $(i,j)$ involved, and hence $N=i\times j$, but then you
would need expresions of $i$ and $j$ in terms of $t$, which are easy
to obtain but makes everything too cumbersome.

Another problem is the matrix $\mathbf{A}$ which suddenly appears here,
without definition.
One might think that $\mathbf{A}$ is just $\mathbf{X}^T$,
but then there are problems with the dimension of the matrices in (3.6).

Summarizing, you need to give a clear presentation for the various
expressions of the functional $J(\mathbf{X})$.

\myitem{P 39, LL 15-19, Eqs. (3.6), (3.7), (3.8)}
Here you start from equation (3.6), $\mathbf{A}\,\mathbf{X}=\mathbf{Y}$,
but you have not defined these matrices, except for their dimensions.

Then you say something about the \red{optimal} solution of (3.6).
What do you mean by \red{optimal} here?

An equation, alone, like (3.6), cannot have an \red{optimal} solution.
If (3.6) has many solutions, then some of them might be optimal from
some point of view, regarding --for instance-- some particular mimisation
or maximisation problems, but they could be not optimal for other
minimax problems.
If (3.6) has no solution, then still one can find some $\mathbf{X}$'s
such that $\mathbf{A}\,\mathbf{X}$ is as close as possible to $\mathbf{Y}$
in some appropriate sense.
These are dubbed \red{pseudo-solutions}.
One of these \red{pseudo-solutions} is obtained via More-Penrose.
This is clear when $\mathbf{X}$ and $\mathbf{Y}$ are vectors.
But what is the situation when $\mathbf{X}$ and $\mathbf{Y}$ are matrices
like here?

Then in (3.7) and (3.8) you talk about $\mathbf{A}^+$ and give the usual
formula when $\mathbf{X}$ and $\mathbf{Y}$ are vectors.
Holds this formula when $\mathbf{X}$ and $\mathbf{Y}$ are matrices

Is all this related to the minimisation problem for $J(\mathbf{X})$?
If yes, you should give proofs!

In an \red{Appendix} you should provide proofs of all claims involving
(3.6), (3.7), (3.8).


\myitem{From P 39, L -7, until P 40, L 3. 
\red{3.2.4 Ridge Regression. THIS IS AN EXPANDED VERSION OF THE LAST
ITEM FROM PART A}}
You present this paragraph as someting called ``Demo''.
What is it? A proof? A proof of what? 
What is your claim before starting with ``Demo''?

\red{It seems that your ``Demo'' has to do with the case when
$\mathbf{A}^T\mathbf{A}$ is not invertible and then you introduce a
regularisation by means of a parameter $\lambda$.
But before your ``Demo'' there is no mention of this fact.
Please clarify!}

You say ``to solve equation (3.8)''.
But (3.8) is just the definition of More-Penrose!

Are you trying to solve (3.6)?
If this is true, what is known and what is not known?
Moreover, what do you mean by \red{solving (3.6)}, specially in the
case (as in your case) when the equation is \red{overdetermined}?

You should prove the equivalence you claim.
At least write $\mathbf{C}_\lambda$ to make explicit the dependence of
$\mathbf{C}$ from the parameter $\lambda$.

I think you should better write something like:
$$
(3.9)\rule{50mm}{0pt} \min_{\mathbf X \in \text{\red{???}}}
\|\mathbf{C}_\lambda\,\mathbf{X}-\mathbf{F}\|
$$
You should make clear in which space $\mathbf{X}$ is running for the
optimisation.

The outcome of this minimisation problem depends on $\lambda$, but
equation (3.6) does not depend on $\lambda$.

Thus, why is (3.6) equivalent to the minimization problem (3.9)?
A proof is needed.

Which value of $\lambda$ will you consider for a solution of (3.6) and why?

What is $\mathbf{A}$ and what is the rationale for introducing $\mathbf{C}$
and $\mathbf{F}$.

The equation for $\mathbf{X}(\lambda)$ in LL 1, 2 in P 40, seems to be a
More-Penrose solution of $\mathbf{C}_\lambda\,\mathbf{X}=\mathbf{F}$.
But why this solution coincides with the solution --if it exists-- of the
minimisation problem?

By the way, is it sure that the minimisation Problem has a solution?
Is it unique? etc.

\vspace{3ex}
We are given
an $m\times n$ matrix of known real coefficients $\mathbf{A}$,
an $n\times l$ matrix of real unknowns $\mathbf{X}$,
an $m\times l$ matrix of real known data $\mathbf{Y}$.

Our problem is to find a matrix $\mathbf{X}$ that minimises the functional
\begin{equation}
J(\mathbf{X}):=\|\mathbf{A}\,\mathbf{X}-\mathbf{Y}\|_2^2\,,
\end{equation}
where $\|\cdot\|_2$ is the Frobenius matrix norm.
Any matrix $\mathbf{X}$ that minimizes $J(\mathbf{X})$ will be called a
{\em minimum\/} of $J(\mathbf{X})$.

\vspace{3ex}
{\bf Claim}. 
1. A minimum $\mathbf{X}_0$ of $J(\mathbf{X})$ always exist. \\
1. If $\mathbf{X}_0$ is a minimum of $J(\mathbf{X})$, then
$\mathbf{X}_0+\mathbf{X}_1$ is also a minimum of $J(\mathbf{X})$
if and only if: 
\begin{equation}
\mathbf{X}_1\in \mathbf{A}^{\perp}:=
\left\{ \mathbf{X}\in\R^n\times\R^l \mid
        \mathbf{A}\,\mathbf{X}=\mathbf{0} \right\}\,.
\end{equation}
3. If the $n\times n$ matrix $\mathbf{A}^{\mathsf{T}}\,\mathbf{A}$ is
invertible and $\mathbf{X}$ and $\mathbf{Y}$ are vectors, then 
the minimum $\mathbf{X}_0$ is given by
$\mathbf{A}^{\mathsf{T}}\,\mathbf{A}$

\end{description}

\vspace{2ex}
Reviewing work in progress/DATE: 31.10.2016

%% \end{document}
