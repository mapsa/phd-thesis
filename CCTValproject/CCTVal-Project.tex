\documentclass[12pt,reqno]{amsart}
\usepackage{amsfonts,amssymb,amsmath,amsthm,mathrsfs,
            amsopn,amsxtra,amstext,amsbsy}
\usepackage{epsfig}
\usepackage{array}
\usepackage{url}
\usepackage{tikz}
%% \usepackage[spanish]{babel}
\usepackage[utf8x]{inputenc}
\usepackage{fancyhdr}
\usepackage{graphicx}
\usepackage{hyperref,wasysym}
\usepackage{xcolor}
\usepackage{lipsum}
\usepackage{enumerate}
\usepackage{enumitem}% http://ctan.org/pkg/enumitem
\usepackage{verbatim}
%% \usepackage[strings]{underscore}

\voffset=-10mm
\oddsidemargin=0pt
\evensidemargin=0pt
\topmargin=-10mm
\headsep=18pt
\headheight=18pt
\footskip=30pt
\topskip=0mm
\textwidth 167mm
\textheight=230mm
\parindent=0em
\marginparsep=0mm
\marginparwidth=0mm
\vfuzz2pt % Don't report over-full v-boxes if over-edge is small
\hfuzz6pt % Don't report over-full h-boxes if over-edge is small


% \newtheorem{conjecture}[theorem]{Conjecture}
\newtheorem{definicion}{Definición}
\newtheorem{theorem}{Theorem}%[section]
% \newtheorem{lemma}[theorem]{Lemma}
% \newtheorem{corollary}[theorem]{Corollary}
% \newtheorem{conjecture}[theorem]{Conjecture}
% \theoremstyle{definition}
% \newtheorem{definition}[theorem]{Definition}
% \newtheorem{example}[theorem]{Example}
% \newtheorem{xca}[theorem]{Exercise}
% \theoremstyle{remark}
% \newtheorem{remark}[theorem]{Remark}
% \numberwithin{equation}{section}

% \def\phi{\varphi}
% \def\epsilon{\varepsilon}
\def\RE{\mbox{\rm Re}\,}
\def\IM{\mbox{\rm Im}\,}
% \def\AZ{|z|}
\newcommand{\rectangle}{\fboxsep0pt\fbox{\rule{0.3em}{0pt}\rule{0pt}{1.5ex}}}

%-------------------------mathcal

\def\A{{\mathcal A }}
\def\CC{{\mathcal C }}
\def\CH{{\mathcal H }}
\def\CK{{\mathcal K }}
\def\CM{{\mathcal M }}
\def\CP{{\mathcal P }}
\def\CS{{\mathcal S }}
\def\CR{{\mathcal R}}
\def\CRu{{\mathcal R}^{\rm u}}
\def\CT{{\mathcal T}}
\def\CTT{\widetilde{{\mathcal T}}}
\def\CW{{\mathcal W}}
\def\Li{{\mbox{\rm Li}}}
\def\WO{{\mathcal W}_{\Omega }}

%-------------------------------Bbb
\def\K{{\mathbb K }}
\def\C{{\mathbb C }}
\def\N{{\mathbb N }}
\def\NO{{\mathbb{N}_0 }}
\def\Z{{\mathbb Z }}
\def\R{{\mathbb R }}
\def\U{{\mathbb U }}
\def\D{{\mathbb D }}
%----------------------------Zusammengesetztes

\def\DB{\overline{\D}}
\def\Db#1{\overline{\D_{#1}}}
\def\DBR{\overline{\D_\rho}}
\def\DV{de la Vall\'ee Poussin }
\def\d#1{\D_{#1}}
\def\H{{\mathcal H}(\D)}
\def\Hb{{\mathcal H}(\DB)}
\def\HO{{\mathcal H}_0(\D )}
\def\Hoo#1{{\mathcal H}(\overline{\D_{#1}})}
\def\HB#1{{\mathcal H}(\D_{#1})}
\def\Ho#1{{\mathcal H}({#1})}
\def\Hyp{{}_2F_1}

\def\achtung#1{\alert{#1}}
\def\blue#1{{\color{blue} #1}}
\def\green#1{{\color{green} #1}}
\def\red#1{{\color{red} #1}}

\def\pick{{\mathscr P}}
\def\logpick{{\text{log}\,{\mathscr P}}}
\def\stammpick{\pick_{\!\!\!\raisebox{-2pt}{\scalebox{0.5}{$\int$}}}}
\def\stammpickpick{\stammpick\cap\pick}
\def\stammpicklogpick{\stammpick\cap\logpick}
\def\expstammpicklogpick{e^{\stammpicklogpick}}


\title[CCTVal Research project proposal]{CCTVal Research project proposal}

\author{Paola Arce}
\author{Werner Kristjanpoller}
\author{Luis Salinas}

\address{CCTVal and Departamento de Inform\'atica, UTFSM, Valpara\'\i so;
         %% \newline
         %% Mathematisches Institut, Universit\"at W\"urzburg;
         \newline
         \today} % October 06, 2016}

%\email{luis.salinas@usm.cl}

\begin{document}

\maketitle%

\def\myitem#1{ \item[\blue{\bf #1}] }

\section{Introduction}
Cointegration is a long-run property of some non-stationary time series where a
linear combination of those time series is stationary.  This behaviour has been
studied in finance because cointegration restrictions often improve forecasting.
The Vector Error Correction Model (VECM)  \cite{engle1987}, is a well-known econometric technique
that characterises short-run variations of a set of cointegrated time series
incorporating long-run relationships as an error correction term. However, the use of VECM has been limited to low frequency time series processed in batch mode mainly due to computational limitations. 

The first computational limitation to address is that VECM parameters are obtained using the ordinary least squares (OLS)
method.  Even though OLS is extensively used, it has over-fitting
issues and it gets computationally expensive when more data is added. In this project we explore the use of Ridge regression which is commonly used
instead of OLS since they include a regularisation parameter which could improve
prediction error. 

The second limitation is the calculation of the cointegration vectors. Our proposal is to avoid their calculation at every step since they represent a long-run relationship between the time series.

For this research project proposal, we will design and evaluate an online version of VECM which optimises how model parameters are obtained so the model can be used with high frequency data. The proposal is an improvement of the online VECM presented in \cite{icpram15} by adding the following modifications:
\begin{itemize}
\item The use of RR instead of OLS to get the model parameters. This will imply the finding of the best regularisation parameter at every time step.
\item The evaluation of the Aggregating Algorithm for Regression (AAR) which is a variation of the online version of Ridge regression.
\item Addition of rules to define when the cointegration vectors must be recalculated.
\item Analysis of different time series frequencies to determine where the model works best.
\item The selection of other relevant data to the model such as bid, ask and volume.
\end{itemize}

The findings of this project could be easily used as an integration tool
to detect arbitrage opportunities or risk control in financial time series. Also, the methodology could be extended to other econometric methods which could be enhanced by machine learning techniques.

\section{Background}

In order to use the VECM, the time series must be integrated of order 1, denoted I(1), i.e. they
become stationary at their first differences. In finance, I(1) time series are
very common. VECM introduces the cointegration relationship as an error correction
term which often improves forecasting, see \cite{duy1998}. Therefore, VECM has been widely adopted in
financial applications \cite{mukherjee1995,seong2013,maysami2000,arestis2001}. VECM has also been used in pair trading, see \cite{herlemont2003}, or models with more than two variables, see
for example \cite{mukherjee1995} and \cite{engle2004}.


Cointegration relationships can be found in low and high frequency data of two or
more assets. While cointegration in low frequency data is motivated by a
long-run equilibrium relationship between economic forces, cointegration in high
frequency data has its foundation in statistical arbitrage theory which is very helpful
to detect mean-reverting trades. Information about cointegrated assets in high
frequency data could be used as an input for high frequency trading strategies,
using it as a signal to capture small profits in short term trades, see
\cite{miao2014}. \cite{zhou2001} addressed the benefits of using higher
frequency data to analyse cointegration. \cite{rittler2012} also found that
cointegration may differ with different data frequencies and could appear
with increased data frequency.

The use of VECM with high frequency data is mainly limited by computationally
expensive routines:
\begin{itemize}
\item VECM parameters are obtained using the ordinary
least squares (OLS) which has over-fitting
issues and is computationally expensive when
the number of lagged values and data increases. This problem is commonly addressed using
Ridge Regression (RR) \cite{hoerl1970}. RR introduces a regularisation
parameter that leads to an unbiased estimation with better generalisation
capability. 
\item The Johansen method is used to obtain 
cointegration vectors and it is also an expensive routine  of order $O(n^3)$  \cite{johansen1995}. A distributed processing approach has been used recently successfully reducing execution times, see \cite{chen2003}, \cite{Arce2017}. Specially in \cite{Arce2017}we showed the advantage of distributed processing over conventional rolling window processing.
\end{itemize}
Therefore, our aim is to design an online method based on cointegration using RR instead of OLS and that works on a distributed environment so it can be used with high frequency stream data. 

\section{Research Design and Methods}

In this proposal, we will extend the approach presented in \cite{icpram15}. In this article we proposed method called OVECM that only considered a time varying window of historical data using OLS to get model parameters. This proposal will explore the use of RR or its variant The  Aggregating Algorithm for Regression (AAR)  \cite{vovk2001} as an alternative of OLS to estimate model parameters.

Additionally, since cointegration vectors represent a long-run relationship between the time
series, they vary little in time. Therefore, we plan to obtain new
cointegration vectors only when this long-term relationship changes, this can be tracked using a threshold of forecast error for example.

We will estimate the performance of our method using different frequencies of financial time series, so we can measure the effect of cointegration in high frequency data.

\section{Hypothesis}

We hypothesise that a regularisation solution could lead to better forecasting performance than OVECM. RR is usually introduced to get better generalisation capabilities but the choice of the regularisation parameter is a challenge to accomplish this goal.

We also hypothesise that, using high frequency data, cointegration vectors will remain the same for some period of time, since they represent long-run relationships.



\bibliographystyle{amscls/amsalpha}
\bibliography{reference}

\end{document}


