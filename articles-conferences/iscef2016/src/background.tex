\def\rot#1{{\color{red}{#1}}}
\def\gruen#1{{\color{green}{#1}}}
\def\blau#1{{\color{blue}{#1}}}


\section{Background}
\label{sec:background}

\subsection{Integration and Cointegration}\label{sec:coint}\  
Following Johansen \citep{johansen1995} we shall say that a stochastic process
$Y_t$ which satisfies $Y_t-E(Y_t) = \sum_{i=0}^\infty C_i\,\varepsilon_{t-i}$ is
called $I(0)$, and then we shall write $Y_t\sim I(0)$, whenever
$\sum_{i=0}^\infty C_i \neq 0$ and $\sum_{i=0}^\infty C_i\,z^i$ converges for
$z\in\mathbb{C}$ with $|z|<1$.  It is understood that the condition
$\varepsilon_t\sim iid(0,\sigma^2)$ holds.

A (vector) time series $\mathbf{y}_t$ is said to be {\em integrated of order\/}
$d$, and then we shall write $\mathbf{y}_t\sim I(d)$, whenever after $d$ times
(discrete) differentiation an stationary process is
obtained \citep{banerjee1993}; more precisely, whenever
$(1-L)^d\,\mathbf{y}_t\sim\text{I(0)}$, where $L$ is the usual lag operator:
$(1-L)\,\mathbf{y}_t = \Delta\mathbf{y}_t = \mathbf{y}_t-\mathbf{y}_{t-1}$ for
all $t$.  

Note that this definition includes the scalar case as time series of vectors of
dimension 1; in this scalar case we will write the time series in nonbold
format.

Let $\mathbf{y}_t^\nu$, $\nu=1,\dots,l$, be a set of $l$ vector time series of
order $I(1)$.  They are said to be {\em cointegrated\/} if a vector
$\beta=[\beta(1),\dots,\beta(l)]^\top \in \mathbb{R}^p$ exists, such that the
time series,
\begin{equation}
\mathbf{Z}_t:= 
\sum_{\nu=1}^l \beta(\nu)\,\mathbf{y}_t^\nu\,\sim\,\text{I(0)}\,.
\end{equation}
In other words, a set of $I(1)$ time series is said to be cointegrated if a
linear combination of them exists, which is I(0).


\subsection{Vector Autorregresive Models}\label{sec:varvec}

Vector error correction model (VECM) describe the joint behaviour of a set of
variables and can be derived from the simple Vector Autoregressive model (VAR)
\citep{sims1980} model.  The VAR($p$) model is a framework describing the
behaviour of a set of $l$ endogenous and stationary variables as a linear
combination of their last $p$ values, where $l,p\in\mathbb{N}$.  In our case,
each one of these $l$ variables is a scalar time series $y_{\lambda,t}$,
$\lambda=1,\dots,l$, and we represent them all together at time $t$ by the
vector time series:
\begin{equation}
\label{eq:variables}
\mathbf{y}_t = 
\begin{bmatrix} y_{1,t} & y_{2,t} & \dots & y_{l,t} \end{bmatrix}^\top.
\end{equation}
\noindent
Notice that the vectors $\mathbf{y}_t$ are assumed to be $l$-dimensional.

The VAR($p$) model describes the behaviour of a dependent variable in terms of
its own lagged values and the lags of the others variables in the system. The
model with $p$ lags is formulated as the system:
\begin{align}
\label{eq:var}
\mathbf{y}_t 
= \boldsymbol{\Phi}_1 \mathbf{y}_{t-1} +
  \boldsymbol{\Phi}_2 \mathbf{y}_{t-2} + \dots +
  \boldsymbol{\Phi}_p\mathbf{y}_{t-p} +
  \mathbf{c} + \boldsymbol{\epsilon}_t \nonumber \\
t=p+1,\dots,N,
\end{align}
\noindent where 
$\boldsymbol{\Phi}_1, \boldsymbol{\Phi}_2,\dots,\boldsymbol{\Phi}_p$
are $l\times l$-matrices of real coefficients,
$\boldsymbol{\epsilon}_{p+1},
 \boldsymbol{\epsilon}_{p+2}, \dots, \boldsymbol{\epsilon}_N$ 
are error terms, $\mathbf{c}$ is a constant vector and $N$ is the total
number of samples.

Notice that, regarding our notation of section (\ref{sec:coint}),
we have here 
$\mathbf{y}_t^0 = \mathbf{y}_t$,
$\mathbf{y}_t^\nu = \mathbf{y}_{t-\nu}$ and
the $\lambda$-th component of the vector time series $\mathbf{y}_t^\nu$
is the scalar time series $y_{\lambda,t}^\nu$, where $\nu=1,\dots,p$ and
$\lambda=1,\dots,l$.

However, the VAR model cannot be used with non-stationary variables. VECM
\citep{engle87} is also a linear model and a special form of a VAR model for I(1)
variables that are also cointegrated \citep{banerjee1993}. If cointegration
exists, variable differences are stationary and they introduce an error
correction term which adjusts coefficients to bring the variables back to
equilibrium. 


It is obtained re-writing equation (\ref{eq:var}) in terms of the new
variable $\Delta\mathbf{y}_t=\mathbf{y}_t-\mathbf{y}_{t-1}$.
The VECM model, expressed in terms those differences, takes the form:
\begin{equation}\label{eq:vec}
\Delta \mathbf{y}_t 
= \boldsymbol{\Omega}\,\mathbf{y}_{t-1}
  + \sum_{i=1}^{p-1} \boldsymbol{\Phi}_i^*\,\Delta\mathbf{y}_{t-i}
  + \mathbf{c} + \boldsymbol{\epsilon}_t\,,
\end{equation}
\noindent
where the coefficients matrices $\boldsymbol{\Phi}_i^*$ and 
$\boldsymbol{\Omega}$, expressed in terms of the matrices
$\boldsymbol{\Phi}_i$ of (\ref{eq:var}), are:
\begin{align*}
\boldsymbol{\Phi}_i^* 
&:= -\sum_{j=i+1}^{p}\boldsymbol{\Phi}_j\,, \\
\boldsymbol{\Omega}
&:= -\left( \mathbb{I} - \boldsymbol{\Phi}_1 - \dots 
    - \boldsymbol{\phi}_p \right)\,. 
\end{align*}
The following well known properties of the matrix $\boldsymbol{\Omega}$
\citep{johansen1995} will be useful in the sequel:
\begin{itemize}
\item
If $\boldsymbol{\Omega} = \mathbf{0}$, there is no cointegration.
\item 
If $rank(\boldsymbol{\Omega})=l$, i.e., if $\boldsymbol{\Omega}$ has
full rank, then the time series are not I(1) but stationary.
\item
If $rank(\boldsymbol{\Omega})=r$, $0<r<l$, then there is cointegration
and the matrix $\boldsymbol{\Omega}$ can be expressed as
$\boldsymbol{\Omega}=\boldsymbol{\alpha\beta}^\top$, where $\boldsymbol{\alpha}$
and $\boldsymbol{\beta}$ are
$l\times r$ matrices and
$\text{rank}(\boldsymbol{\alpha})=\text{rank}(\boldsymbol{\beta})=r$.
\item
The columns of $\boldsymbol{\beta}$ contains the cointegration vectors and the rows of
$\boldsymbol{\alpha}$ correspond with the adjusted vectors. 
$\boldsymbol{\beta}$ is obtained by Johansen procedure~\citep{johansen1988},
whereas $\boldsymbol{\alpha}$ has to be determined as a variable in the VECM.
\end{itemize}
It is worth noticing that the factorization of the matrix
$\boldsymbol\Omega$ is not unique, since for any $r \times r$
non-singular matrix $\mathbf{H}$, $\boldsymbol{\alpha}^*:=\boldsymbol{\alpha}\mathbf{H}$,
and $\boldsymbol{\beta}^*=\boldsymbol{\beta}(\mathbf{H}^{-1})^\top$ we have
$\boldsymbol{\alpha\beta}^\top=\boldsymbol{\alpha}^*(\boldsymbol{\beta}^*)^\top$.
If cointegration exists, then equation (\ref{eq:vec}) can be written
as follows:
\begin{equation}\label{eq:vecfull}
\Delta\mathbf{y}_t 
= \boldsymbol{\alpha\beta}^\top\mathbf{y}_{t-1} 
  + \sum_{i=1}^{p-1}\boldsymbol{\Phi}_i^*\,\Delta\mathbf{y}_{t-i}
  + \mathbf{c} + \boldsymbol{\epsilon}_t\,,
\end{equation}
\noindent
which is a VAR model but for time series differences.


Transposing each equation of the system (\ref{eq:vecfull}) we can write
the VECM($p$) model in block-matrix form as:
\begin{equation}\label{eq:vareq}
\mathbf{B} = 
\mathbf{A} \mathbf{X} + 
\mathbf{E} \, , 
\end{equation}
%
\noindent where $\mathbf{B}$ dimension is $((N-p)\times l)$, $\mathbf{A}$
dimension is $((N-p)\times(r+(p-1)l +1))$, $\mathbf{X}$ dimension is $((r+(p-1)l
+1)\times l)$ and $\mathbf{E}$ dimension is $((N-p)\times l)$:
%
\begin{alignat}{3}
\mathbf{B}
&= \begin{bmatrix}
   \Delta\mathbf{y}_{p+1}^\top \\
   \Delta\mathbf{y}_{p+2}^\top \\
   \vdots \\
   \Delta\mathbf{y}_N^\top
   \end{bmatrix}
&\quad
\mathbf{X}
&= \begin{bmatrix}
   \boldsymbol{\alpha}^\top \\
   \boldsymbol{\Phi}_1^{*\top} \\
   \boldsymbol{\Phi}_2^{*\top} \\
   \vdots \\
   \boldsymbol{\Phi}_{p-1}^{*\top} \\
   \mathbf{c}^\top
   \end{bmatrix}
&\quad
\mathbf{E}
&= \begin{bmatrix}
   \boldsymbol{\epsilon}_{p+1}^\top \\
   \boldsymbol{\epsilon}_{p+2}^\top \\
   \vdots \\
   \boldsymbol{\epsilon}_N^\top \\
   \end{bmatrix}
\end{alignat}
\noindent and 
\begin{align}
\mathbf{A} 
&= \begin{pmat}[{....|}]
   \mathbf{y}_p^\top \boldsymbol{\beta} & \Delta \mathbf{y}_p^\top & \Delta\mathbf{y}_{p-1}^\top & \dots 
                    & \Delta\mathbf{y}_2^\top & 1 \cr
   \mathbf{y}_{p+1}^\top  \boldsymbol{\beta} &\Delta\mathbf{y}_{p+1}^\top & \Delta\mathbf{y}_p^\top & \dots
                       & \Delta\mathbf{y}_3^\top & 1 \cr
   \vdots & \vdots & \vdots & \ddots & \vdots & \vdots \cr
   \mathbf{y}_{N-1}^\top  \boldsymbol{\beta} &\Delta\mathbf{y}_{N-1}^\top & \Delta\mathbf{y}_{N-2}^\top & \dots 
                       & \Delta\mathbf{y}_{N-p-1}^\top & 1 \cr
   \end{pmat}\, .
\label{eq:Amatrix}
\end{align}
Taking into account the error term $\mathbf{E}$, equation~(\ref{eq:vareq}) 
can be solved with respect to $\mathbf{X}$ using the ordinary least
squares (OLS) estimation.
