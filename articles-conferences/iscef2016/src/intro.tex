\section{Introduction}
\label{sec:introduction}
In finance, it is common to find variables with long-run equilibrium
relationships. This is termed cointegration and reflects the idea that some sets
of variables cannot wander too far from each other. Cointegration means that one
or more linear combinations of these variables are stationary even though
individually they are not. Some models, such as the Vector Error Correction
(VECM) \citep{engle87}, take advantage of this property and describe the joint
behaviour of several cointegrated variables. VECM introduces this long-run
relationship among a set of cointegrated time series as an error correction
term. These time series must be integrated of order 1, denoted I(1), i.e. they
become stationary at their first differences. In finance, I(1) time series are
very common and to introduce cointegration restrictions in models often improves
forecasting \citep{duy1998}. Therefore, VECM has been widely adopted in financial
applications (\cite{mukherjee1995}, \cite{seong2013}, \cite{maysami2000} and
\cite{arestis2001}). VECM has also been used in pair trading \citep{herlemont2003} or
models with more than two variables (\cite{mukherjee1995} and \cite{engle2004}).

VECM parameters are obtained using the ordinary least squares (OLS) method
\citep{golub1980}. Since OLS involves many calculations, the parameter estimation
is computationally expensive when the number of lagged values and data
increases. Moreover, obtaining cointegration vectors is also an expensive
routine. This is a main limitation to use VECM with stream data with high
frequency.  Chen and Lung's BVECM \citep{chen2003} addresses the advantage of
distributed processing over conventional rolling window processing. 

Our aims were to study if a parallel version of VECM can be used with high
frequency stream data.  Our approach was to determine adaptively the number of
observations and lags of VECM which maximise cointegration relations in the
past.  This search is computationally expensive because the Johansen method
\citep{johansen1995} is required to find cointegration vectors, which is of order
$O(n^3)$. Therefore, our proposal, called Adaptively Vector Error Correction
(AVECM), is to parallelise this search in order to get new parameters before new
data arrives.  Model effectiveness is focused on out-of-sample forecast rather
than in-sample fitting. This criterion allows AVECM prediction capability to be
expressed rather than just explaining data history.  The forecast capability of
our method was measured using MSE and the Theil's $U$-statistic \citep{theil1966}
widely used in economic forecast. Tests were run using four currency rates: Euro
(EUR) to United States Dollar (USD) (EURUSD), British Pound (GBP) to USD
(GBPUSD), USD to Swiss Franc (CHF) (USDCHF) and USD to Japanese Yen (JPY)
(USDJPY) with a 10-seconds frequency.

This paper is organised as follows: section \ref{sec:background} presents the
VAR and VECM, the AVECM algorithm is presented in section \ref{sec:methodology}.
In section \ref{sec:results} we describe the tests carried on to assess the
accuracy and the execution time of AVECM.  This section also includes a
description of the test data.  Section \ref{sec:conclusions} contains the
conclusions and a discussion of future research.
