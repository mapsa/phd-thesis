\section{\uppercase{Conclusions}}
\label{sec:conclusions}
Cointegration in financial time series has been largely studied and the 
Johansen method is commonly used to obtain cointegration relationships. 
In practice, it has been found that cointegration relations change with time. 
However, model-based cointegration such as VECM assumes that cointegration
remains unchanged in time. 
We empirically showed that the Johansen method is sensitive to the number
of lags but also to the amount of data considered.

Moreover, we introduced the notion of {\em percentage of cointegration\/} and
found that out-of-sample forecast performance MSE is related to the value of
this figure in the last samples.  We used this information to set the model
parameters.  Our proposal AVECM consists of an adaptive algorithm to update VECM
parameters every time that new data is available. These parameters are found by
maximising the percentage of cointegration of the last samples or iterations.

Despite the fact that high frequency Forex data can be spurious, the model performance can be less reliable (and more spurious) relative to the lower frequencies (such as 1 minute or 5 minute intervals) adopted by some other studies. However, the deficiency is offset by gain in accuracy from parallel processing which is capable of searching or examining a much larger state space given the same computational time.

Determining VECM parameters was the most expensive routine and it was run
using parallel processes using MPI which allowed a grid search within a range of
values for $L$ and $p$ to be made.
Tests were done using real currency rates data. 
Our proposal was tested in several scenarios using different times of
the day to show its behaviour related to the opening times of
financial markets.

Results showed that our proposed AVECM improves performance measures by finding
parameters of $L$ and $p$ maximising the percentage of cointegration.
These facts were consistent for different times of the day. 
We found that increasing the number of parameters will always lead to
better performance measures.

The parallel implementation allowed the execution times to be reduced
more than 9 times and therefore a response time was obtain before 10
seconds. Since we used 10-second frequencies we can say that our proposal is suitable
for use in an online context for real applications because response times
were less than this frequency.

For future study, it would be interesting to explore the relationship between
cointegration and performance in order to propose new criteria for
improving VECM parameters. It would also be interesting to include more 
explaining variables such as bid-ask spread and change in volume.

