\section{Introduction}
In finance, it is common to find variables with a long-run and/or a short-run
equilibrium relationship. This relationship is called cointegration and it
reflects the idea of that some set of variables cannot wander too far away from
each other. 
%and also high frequency relationship which avoid arbitrage opportunities
%\footnote{Arbitrage is taking advantage of the price differences in different
%markets}.  
Cointegration means that one or more combinations of these variables is
stationary even though individually they are not.

Some financial models, such as  vector error correction (VEC), take advantage of
this property and describe the joint behaviour of several cointegrated
financial instruments.

VEC is a special case of the vector autorregresive model (VAR). VAR is a linear
model which express future values in terms of its own history and other
variables past values. If cointegration exists, VEC model allows to introduce an
error correction term due to its cointegration estimation error. 

VEC as well as VAR model parameters are obtained using ordinary least squares
method (OLS).   Since OLS involves many calculations, the parameter estimation
method is computationally expensive when the number of past values and
observations considered increases. Moreover, OLS is an ill-posed problem which
admits an infinite number of solutions. 

Ridge regression method (RR) tackles this ill-posed problem and is usually
formulated instead of OLS. RR includes a regularization parameter that leads to
better generalization capability. However, RR is still computationally
expensive dealing with large datasets. Recently, online learning algorithms
have been proposed to solve problems with large data sets because of their
simplicity and their ability of updating the model when new data is available. 

There are several popular online methods such as
perceptron~\cite{rosenblatt58}, passive-aggressive~\cite{crammerETall2006},
stochastic gradient descent~\cite{zhang2004}, aggregating
algorithm~\cite{vovk2001} and the second order
perceptron~\cite{cesa-bianchi2005}.  In~\cite{cesa-bianchi2006}, an in-deph
analysis of online learning is provided.  In particular, the {\em aggregating
algorithm for regression} (AAR) method is a recursive formulation of RR
suitable for using in an online context.

In this paper, we propose an online formulation of the VEC model based on the
AAR method considering only a sliding window of the most recent data. The
algorithm introduces matrix optimizations in order to reduce the number of
operations. Our method is later tested with financial data from the foreign
exchange market.

%The next sections are organized as follows: section~\ref{sec:varvec} presents
%the VAR and VEC model, section~\ref{sec:RR} presents the RR formulation,
%section~\ref{sec:ORR} shows the online version of RR, the online VEC algorithm
%proposed is presented in section~\ref{sec:proposal}, section~\ref{sec:data}
%gives a description of the data used, section~\ref{sec:experiments} shows the
%tests carried on to show a comparision in accuracy and time of our proposal and
%section~\label{sec:conclusion} concludes.  
