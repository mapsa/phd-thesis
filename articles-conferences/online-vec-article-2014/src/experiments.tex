\section{Experiments} \label{sec:experiments}

In order to determine the parameter for sliding window size $L$  and
the number of lags, we define a set of base currencies:
EURUSD,GBPUSD,USDJPY,USDCHF and add an extra currency of the rest of
currencies available. The base currencies were chosen because they are
the most transacted currencies.

Later, we run VEC model with the 5 currencies considering a variable
window size data (from 100 to 4000) and number of lags (from 1 to 20)
and determine its akaike information criterion (AIC). The figure
~\ref{fig:lagoptimization} shows the zones with AIC levels varying in
color, were white represent the higher values and red represents the
lower values (best models have lower AIC).  We can see that more red
zones are related with high number of lags and windows size around
3000 in 4 examples. The situation is similar in

\begin{figure}[H]
\centering
\begin{tabular}{cc}

\subfloat[Base currencies + AUDSGD]{
    \includegraphics[width=0.4\linewidth]{img/AIC_AUDSGD}
    \label{fig:currency1}
}
&
\subfloat[Base currencies + NZDJPY]{
    \includegraphics[width=0.4\linewidth]{img/AIC_NZDJPY}
    \label{fig:currency2}
}
\\
\subfloat[Base currencies + AUDNZD]{
    \includegraphics[width=0.4\linewidth]{img/AIC_AUDNZD}
    \label{fig:currency3}
}
&
\subfloat[Base currencies + CHFJPY]{
    \includegraphics[width=0.4\linewidth]{img/AIC_CHFJPY}
    \label{fig:currency4}
}

\end{tabular}
\caption{AIC: Lags vs Sliding window size}
\label{fig:lagoptimization}
\end{figure}

 
