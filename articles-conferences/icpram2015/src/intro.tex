\section{\uppercase{Introduction}}
\label{sec:introduction}
\noindent In finance, it is common to find variables with long-run equilibrium
relationships. This is called cointegration and it reflects the idea of that
some set of variables cannot wander too far from each other. Cointegration means
that one or more linear combinations of these variables are stationary even
though individually they are not~\cite{engle87}. Furthermore, the number of
cointegration vectors reflects how many of these linear combinations exist. Some
models, such as the Vector Error Correction (VECM), take advantage of this property
and describe the joint behaviour of several cointegrated variables.

VECM introduces this long-run relationship
among a set of cointegrated variables as an error correction term. VECM is a
special case of the vector autorregresive model (VAR) model. VAR model
expresses future values as a linear combination of variables past values.
However, VAR model cannot be used with non-stationary variables. VECM is a
linear model but in terms of variable differences. If cointegration exists,
variable differences are stationary and they introduce an error correction term
which adjusts coefficients to bring the variables back to equilibrium. In
finance, many economic time series are revealed to be stationary when they are
differentiated and cointegration restrictions often improves
forecasting~\cite{duy1998}. Therefore, VECM has been widely adopted.

In finance, pair trading is a very common example of cointegration
application~\cite{herlemont2003} but cointegration can also be extended to a
larger set of variables~\cite{mukherjee1995},\cite{engle2004}.

Both VECM and VAR model parameters are obtained using ordinary least squares
(OLS) method. Since OLS involves many calculations, the parameter estimation
method is computationally expensive when the number of past values and
observations increases. Moreover, obtaining cointegration vectors is also an
expensive routine.

Recently, online learning algorithms have been proposed to solve problems with
large data sets because of their simplicity and their ability to update the
model when new data is available. The study presented by ~\cite{arce+salinas2012}
applied this idea using ridge regression.

There are several popular online methods such as perceptron~\cite{rosenblatt58},
passive-aggressive~\cite{crammerETall2006}, stochastic gradient
descent~\cite{zhang2004}, aggregating algorithm~\cite{vovk2001} and the second
order perceptron~\cite{cesa-bianchi2005}.  In~\cite{cesa-bianchi2006}, an
in-deph analysis of online learning is provided. 

In this paper, we propose an online formulation of the VECM called Online VECM
(OVECM). OVECM is a lighter version of VECM which considers only a sliding window
of the most recent data and introduces matrix optimizations in order to reduce
the number of operations and therefore execution times. OVECM also takes into
account the fact that cointegration vector space doesn't experience large
changes with small changes in the input data. 

OVECM is later compared against VECM and ARIMA models using four currency rates
from the foreign exchange market with 1-minute frequency. VECM and ARIMA models
were used in an iterative way in order to allow fair comparison. Execution times
and forecast performance measures MAPE, MAE and RMSE were used to compare all
methods. 

Model effectiveness is focused on out-of-sample forecast rather than in-sample
fitting. This criteria allows the OVECM prediction capability to be expressed
rather than just explaining data history.

The next sections are organized as follows: section~\ref{sec:background}
presents the VAR and VECM, the OVECM algorithm proposed is presented in
section~\ref{sec:methodology}. Section~\ref{sec:results} gives a description of
the data used and the tests carried on to show accuracy and time comparison of
our proposal against the traditional VECM and
section~\ref{sec:conclusions} includes conclusions and a proposal for future study.
