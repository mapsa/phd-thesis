\section{\uppercase{Background}}
\label{sec:background}
\noindent

\subsection{Integration and Cointegration}\label{sec:coint}\  
Following Johansen \cite{johansen1995} we shall say that a stochastic process
$Y_t$ which satisfies $Y_t-E(Y_t) = \sum_{i=0}^\infty C_i\,\varepsilon_{t-i}$ is
called $I(0)$, and then we shall write $Y_t\sim I(0)$, whenever
$\sum_{i=0}^\infty C_i \neq 0$ and $\sum_{i=0}^\infty C_i\,z^i$ converges for
$z\in\mathbb{C}$ with $|z|<1$.  It is understood that the condition
$\varepsilon_t\sim iid(0,\sigma^2)$ holds.

A (vector) time series $\mathbf{y}_t$ is said to be {\em integrated of order\/}
$d$, and then we shall write $\mathbf{y}_t\sim I(d)$, whenever after $d$ times
(discrete) differentiation an stationary process is obtained~\cite{banerjee1993};
more precisely, whenever
$(1-L)^d\,\mathbf{y}_t\sim\text{I(0)}$, where $L$ is the usual lag operator:
$(1-L)\,\mathbf{y}_t = \Delta\mathbf{y}_t = \mathbf{y}_t-\mathbf{y}_{t-1}$ for
all $t$.  

Note that this definition includes the scalar case as time series of
vectors of dimension 1; in this scalar case we will write the time series in
nonbold format.

Let $\mathbf{y}_t^\nu$, $\nu=1,\dots,l$, be a set of $l$ vector time series of
order $I(1)$.  They are said to be {\em cointegrated\/} if a vector
$\beta=[\beta(1),\dots,\beta(l)]^\top \in \mathbb{R}^p$ exists, such that the
time series,
\begin{equation}
\mathbf{Z}_t:= 
\sum_{\nu=1}^l \beta(\nu)\,\mathbf{y}_t^\nu\,\sim\,\text{I(0)}\,.
\end{equation}
In other words, a set of $I(1)$ time series is said to be cointegrated if a
linear combination of them exists, which is I(0).


\subsection{Vector Autorregresive Models}\label{sec:varvec}

VECM is a special case of the VAR \cite{sims1980} model and both describe the joint behaviour
of a set of variables.

The VAR($p$) model is a general framework describing the behaviour of a
set of $l$ endogenous variables as a linear combination of their last $p$
values, where $l,p\in\mathbb{N}$. 
In our case, each one of these $l$ variables is a scalar time series
$y_{\lambda,t}$, $\lambda=1,\dots,l$, and we represent them all together
at time $t$ by the the vector time series:
\begin{equation}
\label{eq:variables}
\mathbf{y}_t = 
\begin{bmatrix} y_{1,t} & y_{2,t} & \dots & y_{l,t} \end{bmatrix}^\top.
\end{equation}
\noindent
Notice that the vectors $\mathbf{y}_t$ are assumed to be $l$-dimensional.

The VAR($p$) model describes the behaviour of a dependent variable in terms of
its own lagged values and the lags of the others variables in the system. The
model with $p$ lags is formulated as the system of $N$:
\begin{align}
\label{eq:var}
\mathbf{y}_t 
= \boldsymbol{\Phi}_1 \mathbf{y}_{t-1} +
  \boldsymbol{\Phi}_2 \mathbf{y}_{t-2} + \dots +
  \boldsymbol{\Phi}_p\mathbf{y}_{t-p} +
  \mathbf{c} + \boldsymbol{\epsilon}_t \nonumber \\
t=p+1,\dots,N,
\end{align}
\noindent where 
$\boldsymbol{\Phi}_1, \boldsymbol{\Phi}_2,\dots,\boldsymbol{\Phi}_p$
are $l\times l$-matrices of real coefficients,
$\boldsymbol{\epsilon}_{p+1},
 \boldsymbol{\epsilon}_{p+2}, \dots, \boldsymbol{\epsilon}_N$ 
are error terms, $\mathbf{c}$ is a constant vector and $N$ is the total
number of samples.

Notice that, regarding our notation of section (\ref{sec:coint}),
we have here 
$\mathbf{y}_t^0 = \mathbf{y}_t$,
$\mathbf{y}_t^\nu = \mathbf{y}_{t-\nu}$ and
the $\lambda$-th component of the vector time series $\mathbf{y}_t^\nu$
is the scalar time series $y_{\lambda,t}^\nu$, where $\nu=1,\dots,p$ and
$\lambda=1,\dots,l$.

However, the VAR model cannot be used with non-stationary variables. VECM \cite{engle87} is also a linear model and a special form of a VAR model for I(1) variables that are also
cointegrated~\cite{banerjee1993}. If cointegration exists, variable differences
are stationary and they introduce an error correction term which adjusts
coefficients to bring the variables back to equilibrium. 


It is obtained re-writing equation (\ref{eq:var}) in terms of the new
variable $\Delta\mathbf{y}_t=\mathbf{y}_t-\mathbf{y}_{t-1}$.
The VECM model, expressed in terms those differences, takes the form:
\begin{equation}\label{eq:vec}
\Delta \mathbf{y}_t 
= \boldsymbol{\Omega}\,\mathbf{y}_{t-1}
  + \sum_{i=1}^{p-1} \boldsymbol{\Phi}_i^*\,\Delta\mathbf{y}_{t-i}
  + \mathbf{c} + \boldsymbol{\epsilon}_t\,,
\end{equation}
\noindent
where the coefficients matrices $\boldsymbol{\Phi}_i^*$ and 
$\boldsymbol{\Omega}$, expressed in terms of the matrices
$\boldsymbol{\Phi}_i$ of (\ref{eq:var}), are:
\begin{align*}
\boldsymbol{\Phi}_i^* 
&:= -\sum_{j=i+1}^{p}\boldsymbol{\Phi}_j\,, \\
\boldsymbol{\Omega}
&:= -\left( \mathbb{I} - \boldsymbol{\Phi}_1 - \dots 
    - \boldsymbol{\phi}_p \right)\,. 
\end{align*}
The following well known properties of the matrix $\boldsymbol{\Omega}$
\cite{johansen1995} will be useful in the sequel:
\begin{itemize}
\item
If $\boldsymbol{\Omega} = \mathbf{0}$, there is no cointegration.
\item 
If $rank(\boldsymbol{\Omega})=l$, i.e., if $\boldsymbol{\Omega}$ has
full rank, then the time series are not I(1) but stationary.
\item
If $rank(\boldsymbol{\Omega})=r$, $0<r<l$, then there is cointegration
and the matrix $\boldsymbol{\Omega}$ can be expressed as
$\boldsymbol{\Omega}=\boldsymbol{\alpha\beta}^\top$, where $\boldsymbol{\alpha}$
and $\boldsymbol{\beta}$ are
$l\times r$ matrices and
$\text{rank}(\boldsymbol{\alpha})=\text{rank}(\boldsymbol{\beta})=r$.
\item
The columns of $\boldsymbol{\beta}$ contains the cointegration vectors and the rows of
$\boldsymbol{\alpha}$ correspond with the adjusted vectors. 
$\boldsymbol{\beta}$ is obtained by Johansen procedure~\cite{johansen1988},
whereas $\boldsymbol{\alpha}$ has to be determined as a variable in the VECM.
\end{itemize}
It is worth noticing that the factorization of the matrix
$\boldsymbol\Omega$ is not unique, since for any $r \times r$
non-singular matrix $\mathbf{H}$, $\boldsymbol{\alpha}^*:=\boldsymbol{\alpha}\mathbf{H}$,
and $\boldsymbol{\beta}^*=\boldsymbol{\beta}(\mathbf{H}^{-1})^\top$ we have
$\boldsymbol{\alpha\beta}^\top=\boldsymbol{\alpha}^*(\boldsymbol{\beta}^*)^\top$.
If cointegration exists, then equation (\ref{eq:vec}) can be written
as follows:
\begin{equation}\label{eq:vecfull}
\Delta\mathbf{y}_t 
= \boldsymbol{\alpha\beta}^\top\mathbf{y}_{t-1} 
  + \sum_{i=1}^{p-1}\boldsymbol{\Phi}_i^*\,\Delta\mathbf{y}_{t-i}
  + \mathbf{c} + \boldsymbol{\epsilon}_t\,,
\end{equation}
\noindent
which is a VAR model but for time series differences.


Transposing each equation of the system (\ref{eq:vecfull}) we can write
the VECM($p$) model in block-matrix form as:
\begin{equation}\label{eq:vareq}
\mathbf{B} = 
\mathbf{A} \mathbf{X} + 
\mathbf{E} \, , 
\end{equation}
%
\noindent where $\mathbf{B}$ dimension is $((N-p)\times l)$, $\mathbf{A}$
dimension is $((N-p)\times(r+(p-1)l +1))$, $\mathbf{X}$ dimension is $((r+(p-1)l
+1)\times l)$ and $\mathbf{E}$ dimension is $((N-p)\times l)$:
%
\begin{alignat}{3}
\mathbf{B}
&= \begin{bmatrix}
   \Delta\mathbf{y}_{p+1}^\top \\
   \Delta\mathbf{y}_{p+2}^\top \\
   \vdots \\
   \Delta\mathbf{y}_N^\top
   \end{bmatrix}
&\quad
\mathbf{X}
&= \begin{bmatrix}
   \boldsymbol{\alpha}^\top \\
   \boldsymbol{\Phi}_1^{*\top} \\
   \boldsymbol{\Phi}_2^{*\top} \\
   \vdots \\
   \boldsymbol{\Phi}_{p-1}^{*\top} \\
   \mathbf{c}^\top
   \end{bmatrix}
&\quad
\mathbf{E}
&= \begin{bmatrix}
   \boldsymbol{\epsilon}_{p+1}^\top \\
   \boldsymbol{\epsilon}_{p+2}^\top \\
   \vdots \\
   \boldsymbol{\epsilon}_N^\top \\
   \end{bmatrix}
\end{alignat}
\noindent and 
\begin{align}
\mathbf{A} 
&= \begin{pmat}[{....|}]
   \mathbf{y}_p^\top \boldsymbol{\beta} & \Delta \mathbf{y}_p^\top & \Delta\mathbf{y}_{p-1}^\top & \dots 
                    & \Delta\mathbf{y}_2^\top & 1 \cr
   \mathbf{y}_{p+1}^\top  \boldsymbol{\beta} &\Delta\mathbf{y}_{p+1}^\top & \Delta\mathbf{y}_p^\top & \dots
                       & \Delta\mathbf{y}_3^\top & 1 \cr
   \vdots & \vdots & \vdots & \ddots & \vdots & \vdots \cr
   \mathbf{y}_{N-1}^\top  \boldsymbol{\beta} &\Delta\mathbf{y}_{N-1}^\top & \Delta\mathbf{y}_{N-2}^\top & \dots 
                       & \Delta\mathbf{y}_{N-p-1}^\top & 1 \cr
   \end{pmat}\, .
\label{eq:Amatrix}
\end{align}
Taking into account the error term $\mathbf{E}$, equation~(\ref{eq:vareq}) 
can be solved with respect to $\mathbf{X}$ using the ordinary least
squares (OLS) estimation.

\subsection{Ordinary Least Squares method}

When $\mathbf{A}$ is singular, solution to equation~(\ref{eq:vareq}) is given
by the ordinary least squares (OLS) method. OLS consists of minimizing the sum
of squared errors or equivalently minimizing the following expression:

\begin{equation}
\label{eq:regressionproblem}
\underset{\mathbf{X}}{\text{min}} \quad \| \mathbf{A}\mathbf{\mathbf{X}} - \mathbf{B} \|_2^2
\end{equation}

\noindent for which the solution $\hat{\mathbf{X}}$ is well-known:

\begin{equation}
\label{eq:MP}
\hat{\mathbf{X}}=\mathbf{A}^{\!\!+}\,\mathbf{B}
\end{equation}

\noindent where $\mathbf{A}^{\!\!+}$ is the Moore-Penrose pseudo-inverse
which can be written as follows: 

\begin{equation}
\label{eq:pseudoinverse}
\mathbf{A}^{\!\!+}= (\mathbf{A}^{\!\!\top} \mathbf{A})^{-1}\mathbf{A}^{\!\!\top} \, .
\end{equation}

However, when $\mathbf{A}$ is not full rank, i.e
$rank(\mathbf{A})=k <  n \leq m$, $\mathbf{A}^\top \mathbf{A}$ is
always singular and equation~(\ref{eq:pseudoinverse}) cannot be used.
More generally, the pseudo-inverse is best computed using the compact
singular value decomposition (SVD) of $\mathbf{A}$:

\begin{equation}
    \label{eq:compactsvd}
    \underset{m \times n}{\mathbf{A}}=
    \underset{m \times k}{\mathbf{U_1}} \enskip
    \underset{k \times k}{\Sigma_1} \enskip
    \underset{k \times n}{\mathbf{V}_1^{\top}} \, ,
\end{equation}

\noindent as follows

\begin{equation}
\label{eq:pseudoinversesvd}
\mathbf{A}^{\!\!+} = \mathbf{V}_1 \Sigma_1^{-1} \mathbf{U}_1^\top \, .
\end{equation}



\subsection{Regularization}\label{sec:RR}

Regularization can be understood using two contexts: learning theory (probabilistic) and inverse problems (deterministic). In the context of learning, regularization refers to techniques allowing to avoid over-fitting and the desire property of the selected estimator is to perform well on new data (to generalize), e.g. regularized least squares. In the context of inverse problems, regularization objective is to stabilise, with respect to noise, a possibly ill-conditioned matrix inversion problem e.g spectral cut-off and Tikhonov regularization or ridge regression (RR)\cite{ tikhonov1977}.

In particular, it is well known that regularization schemes such as RR or Tikhonov regularization can be effectively used in the context of learning \cite{vito2005}.

Ridge regression was independently proposed by Tikhonov \cite{tikhonov1963} and Phillips \cite{phillips1962}. RR penalise the complexity of the  solution by adding a constraint with respect to OLS (see equation \ref{eq:regressionproblem}) in the optimisation problem:

\begin{equation}
\label{eq:RRproblem}
\underset{\mathbf{X}}{\text{min}} \quad \|
\mathbf{A}\mathbf{\mathbf{X}} - \mathbf{B} \|_2^2 +\lambda \|
\mathbf{\mathbf{X}}\|_2^2 
\end{equation}

\noindent where $\lambda>0$ is a regularization parameter. The RR optimal solution $\mathbf{X}(\lambda)$ is also well-known: 

\begin{equation}
\label{eq:optsolRR}
\mathbf{X}(\lambda)=(\mathbf{A}^\top \mathbf{A}+ \lambda
\mathbb{I})^{-1}\mathbf{A}^\top \mathbf{B} \, . 
\end{equation}

The method is called ridge regression because the term $\lambda \mathbb{I}$ adds positive entries along the diagonal (ridge) to avoid the
singularity of the covariance matrix $\mathbf{A}^\top \mathbf{A}$. This addition ensures that all of the covariance matrix eigenvalues will be strictly greater than 0, i.e the solution becomes unique.

Despite the fact that RR is a biased estimator (OLS is unbiased) it could
reduce the expected prediction error by reducing variance, this is called the bias-variance tradeoff  \cite{hoerl1970}. 

To determine parameter $\lambda$ is crucial for ridge regression, but the bias-variance tradeoff is not directly useful to determine it, since que don't know the data we want to predict. Many techniques have been developed to determine suitable value for the regularization parameter. However, when no information is available on the error, one of the most popular techniques is the Generalized Cross Validation (GCV) \cite{bauer2011}. 
GCV originates from the older method leave-one-out Cross Validation \cite{stone1974}.

The leave-one-out cross validation method determine a regularization parameter that minimises the average of the squared predictions errors using each solution to predict the missing data value.

%'---------------------------http://robjhyndman.com/hyndsight/crossvalidation/
%When the data are not independent cross-validation becomes more difficult as leaving out an observation does not remove all the associated information due to the correlations with other observations. For time series forecasting, a cross-validation statistic is obtained as follows

%Fit the model to the data y1,…,yt and let y^t+1 denote the forecast of the next observation. Then compute the error (et+1∗=yt+1−y^t+1) for the forecast observation.
%Repeat step 1 for t=m,…,n−1 where m is the minimum number of observations needed for fitting the model.
%Compute the MSE from em+1∗,…,en∗.
%References

%An excellent and comprehensive recent survey of cross-validation results is Arlot and Celisse (2010)

%---------------------------------------------------
\subsection{Online machine learning algorithms}

An online algorithm allows incremental learning by processing one instance at a
time. This is done updating the current model instead of building the model from
scratch.

Training phase is commonly a computationally expensive process. Therefore, when
new data arrives it can't be included easily to the model. Moreover, it could
happen that we won't have enought time to process the new data before more data
arrives. In online learning algorithms there is no training phase, but the model
is updated and evaluated at every time step. This model updating is
computationally less expensive than a training phase.

Online learning is specially useful for stream data problems and the model is
sensitive to new data. It is also useful when some past data may be irrelevant
or we want to improve computational efficiency. However, online algorithms could
affect accuracy.

There are several popular online methods such as
perceptron~\cite{rosenblatt58},winnow~\cite{littlestone1988},
passive-aggressive~\cite{crammerETall2006}, stochastic gradient
descent~\cite{zhang2004}, aggregating algorithm~\cite{vovk2001} and the second
order perceptron~\cite{cesa-bianchi2005}.  In \cite{blum1998} and
\cite{cesa-bianchi2006} an in-deph analysis of online learning is provided.
Applications in finance has been widely used: study presented
by~\cite{arce+salinas2012} applied ridge regression in an online context and
more recently, time series forecasting using online learning has been
presented~\cite{anavaetAl2013}.

Incremental learning refers to any online learning process that learns the same
model as would be learnt by a batch learning algorithm. 

Incremental learning is useful when the input to a learning process is stream
data, with the need or desire to be able to use the result of learning at any
point in time, based on the input observations received so far. 

In principle, the stream of observations may be infinitely long, or the next
observation long delayed, precluding any hope of waiting until all the
observations have been received. One would rather not simply accumulate and
store all the inputs and, upon receipt of each new one, apply a batch learning
algorithm to the entire sequence of inputs received so far. It would be
preferable compu- tationally if the existing hypothesis or other artifact of
learning could be updated in response to each newly received input observation.

Incremental learning is very useful when there is no need to record fundamental
data and only a summary needs to be retained. Due to this, incremental
algorithms are often characterized as memoryless, because no memory of past data
is required.  The algorithm is online but not incremental if it doesn't produce
the same result for all observations that the corresponding batch algorithm
would for these same observations.

Algorithm~\ref{alg:onlinealg} shows the online learning algorithm structure:

\begin{algorithm}[ht]
\begin{algorithmic}[1]
    \STATE Receives input $\mathbf{x}_t$
    \STATE Makes prediction $\mathbf{\hat{y}}_t$
    \STATE Receives response $\mathbf{y}_t$
    \STATE Incurs loss $l_t(\mathbf{y}_t,\mathbf{\hat{y}}_t)$
\end{algorithmic}
\caption{Structure of a Learning System}
\label{alg:onlinealg}
\end{algorithm}

\noindent where $l$ is some loss function. Performance is later measured after
$T$ trials as:

\begin{equation*}
L_T = \sum_{t=1}^T l_t(\mathbf{y}_t,\mathbf{\hat{y}}_t)
\end{equation*}

The objective is minimize this loss function for all instances.
The quality of online learning algorithms is measured by a quantity known as
regret which is the difference between the performance of the online algorithm
and its optimal predictor $E^* \in \Theta$ given by:

\begin{equation*}
L^*_T= \text{min}_{E \in \Theta} L_T^E \, ,
\end{equation*}

\noindent where $L_T^E = \sum_{t=1}^T
l_t(\mathbf{y}_t,\mathbf{y}^E_t)$ and $\mathbf{y}^E_t$ is the expert estimation. 

Therefore regret is defined as:

\begin{equation*}
R_T = L_T - L^*_T
\end{equation*}


%An online learning algorithm looks at every example exactly once.
\subsection{Model selection}
In order to set model parameters the Akaike Information Criterion (AIC) is
used. AIC is often used in model selection where AIC whith smaller values are
preferred.

AIC is calculated as follows:

\begin{equation}
\label{eq:aicformula}
AIC = \underset{\text{bias}}{-\frac{2l}{N}} + 
\underset{\text{variance}}{\frac{2k}{N}}
\end{equation}

\noindent where 

\begin{description}
\item[l] is the loglikelihood function
\item[k] number of estimated parameters (including the variance)
\item[N] number of observations
\end{description}

\subsection{Online Ridge regression}

If we represent matrices $\mathbf{A}$ and $\mathbf{B}$ such as:

\begin{equation}
\label{eq:notation}
	\mathbf{A} = 
\left[
  \begin{tabular}{c>{$}c<{$}c}
    --- & \mathbf{a}^{\top}_{1} & ---\\
    --- & \mathbf{a}^{\top}_{2} & ---\\
    & \vdots & \\
    --- & \mathbf{a}^{\top}_{m} & ---
  \end{tabular}
\right]
\quad \text{and} \quad
\mathbf{B} =
\left[
  \begin{tabular}{c>{$}c<{$}c}
    --- & \mathbf{b}_{1} & ---\\
    --- & \mathbf{b}_{2} & ---\\
    & \vdots & \\
    --- & \mathbf{b}_{m} & ---
  \end{tabular}
\right] \, ,
\
\end{equation}

\noindent equation~(\ref{eq:optsolRR}) can also be written as:

\begin{eqnarray*}
\label{eq:RReapand}
\mathbf{\mathbf{X}}_{\text{ridge}}&=&(\mathbf{A}^\top \mathbf{A}+ \lambda
\mathbb{I})^{-1}\mathbf{A}^\top \mathbf{B} \\
&=& \displaystyle \big (\sum_{t=1}^m
\mathbf{a}_t \mathbf{a}_t  ^\top + \lambda \mathbb{I}\big )^{-1}
\sum_{t=1}^m \mathbf{a}_t \mathbf{b}_t \, .
\end{eqnarray*}

Lets define $\displaystyle\mathbf{S}= \sum_{t=1}^m \mathbf{a}_t
\mathbf{a}_t  ^\top + \lambda \mathbb{I} $ and $\mathbf{W}=
\displaystyle\sum_{t=1}^m \mathbf{a}_t \mathbf{b}_t$, so the
algorithm~\ref{alg:RR} shows the iterative formulation:

\begin{algorithm}[H]
\begin{algorithmic}[1]
\REQUIRE $\,$ \\
$\{\mathbf{a}_1,\dots,\mathbf{a}_{m} \}$: $m$ input vectors \\
$\{\mathbf{b}_1,\dots,\mathbf{b}_{m} \}$: $m$ target vectors \\
$\lambda$: regularization parameter \\
\ENSURE  $\,$ \\
$\{f(\mathbf{a}_1),\dots,f(\mathbf{a}_{m}) \}$: model predictions \\
\STATE Initialize $\mathbf{S}=\lambda \mathbb{I}$
and $\mathbf{W}=0$
\FOR { $t = 1$ to $m$ }
	\STATE read new $\mathbf{a}_t$
	\STATE $\mathbf{X}=\mathbf{S}^{-1}\mathbf{W}$
	\STATE output prediction $f(\mathbf{a}_t) = \mathbf{X}^\top \mathbf{a}_t$
   	\STATE $\mathbf{S} = \mathbf{S} + \mathbf{a}_t \mathbf{a}_t^\top$
   	\STATE Read new $y_t$
    	\STATE $\mathbf{W} = \mathbf{W} + \mathbf{a}_t \mathbf{b}_t$
\ENDFOR
\end{algorithmic}
\caption{Online Ridge Regression}
\label{alg:RR}
\end{algorithm}



\subsection{The Aggregating Algorithm for Regression}

The AAR, proposed by~\cite{vovk2001}, is an application of the aggregating
algorithm to the problem of regression. The idea is to introduce the new input
vector $\mathbf{x}_{m+1}$ to solve the model parameters: 

\begin{equation}
\label{eq:AARexpand}
\mathbf{X}_{aar} = \displaystyle \big (\sum_{t=1}^{m+1}
\mathbf{a}_t \mathbf{a}_t  ^\intercal + \gamma \mathbb{I}\big )^{-1}
\sum_{t=1}^m \mathbf{a}_t \mathbf{b}_t \, .
\end{equation}

If we define $\displaystyle\mathbf{S}= \sum_{t=1}^{m+1} \mathbf{a}_t
\mathbf{a}_t  ^\intercal + \gamma \mathbb{I} $ and $\mathbf{W}=
\displaystyle\sum_{t=1}^m \mathbf{a}_t \mathbf{b}_t$, the
algorithm~\ref{alg:AAR} is slightly different to the algorithm~\ref{alg:RR}, 
which updated matrix $\mathbf{S}$ before making the prediction:

\begin{algorithm}[ht]
\begin{algorithmic}[1]
\REQUIRE $\,$ \\
$\{\mathbf{a}_1,\dots,\mathbf{a}_{m} \}$: $m$ input vectors \\
$\{\mathbf{b}_1,\dots,\mathbf{b}_{m} \}$: $m$ target vectors \\
$\lambda$: regularization parameter \\
\ENSURE  $\,$ \\
$\{f(\mathbf{a}_1),\dots,f(\mathbf{a}_{m}) \}$: model predictions \\
\STATE Initialize $\mathbf{S}=\lambda \mathbb{I}$
and $\mathbf{W}=0$
\FOR { $t = 1$ to $m$ }
	\STATE read new $\mathbf{a}_t$
   	\STATE $\mathbf{S} = \mathbf{S} + \mathbf{a}_t \mathbf{a}_t^\intercal$
	\STATE $\mathbf{X}=\mathbf{S}^{-1}\mathbf{W}$
	\STATE output prediction $f(\mathbf{a}_t) = \mathbf{X}^\intercal \mathbf{a}_t$
   	\STATE Read new $\mathbf{y}_t$
    	\STATE $\mathbf{W} = \mathbf{W} + \mathbf{a}_t \mathbf{b}_t$
\ENDFOR
\end{algorithmic}
\caption{{\em The aggregating algorithm for regression}}
\label{alg:AAR}
\end{algorithm}
