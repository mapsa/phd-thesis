\section{\uppercase{Introduction}}
\label{sec:introduction}
\noindent 

Cointegration concept was introduced by~\cite{engle1987} and implies that one or
more linear combinations of non-stationary variables are stationary even though
individually they are not.  Moreover ~\cite{stock+watson1988} observed that
cointegration reflects the common stochastic trends providing a useful way to
understand cointegration relationships. These common stochastic trends can be
also interpreted as a long-run equilibrium relationships.

This idea was inmediatly adopted in finance since it could represent their
long-run relationship implied by economic theory~\cite{laietAl1991},
\cite{lence+falk2005}.  Economic theory suggest that economic time series are
mean-reverting process and therefore, it reflects the idea of that some set of
variables cannot wander too far from each other. 

On the other hand, the efficient markets hypothesis, also known as the random
walk theory states that current stock prices fully reflect available
information related to its value and there is no way to earn excess
profits~\cite{fama1970}.  This means that if we have stock prices from a
jointly efficient market, they cannot be
cointegrated~\cite{granger1986},\cite{dwyer1992}. However, \cite{richards1995}
claims that cointegration is directly at odds with market efficiency, even
though, there is no evidende that cointegration among asset prices have
implications about market efficiency~\cite{lence+falk2005}.

Despite the fact that cointegration on closing daily rates of currency pairs has
not been found \cite{coleman1990}, \cite{copeland1991}, different time series
frequencies can have different behaviours~\cite{aldridge2009}. Pair trading is a
very common example of cointegration application~\cite{herlemont2003} 
but cointegration can also be extended to a larger set of
variables~\cite{mukherjee1995},\cite{engle2004}.

%So why do we care about cointegration? Someone else can probably give more
%econometric applications, but in quantitative finance, cointegration forms the
%basis of the pairs trading strategy: suppose we have two cointegrated stocks X
%and Y, with the particular (for concreteness) cointegrating relationship X - 2Y
%= Z, where Z is a stationary series of zero mean. For example, X could be
%McDonald's, Y could be Burger King, and the cointegration relationship would
%mean that X tends to be priced twice as high as Y, so that when X is more than
%twice the price of Y, we expect X to move down or Y to move up in the near
%future (and analogously, if X is less than twice the price of Y, we expect X to
%move up or Y to move down). This suggests the following trading strategy: if X -
%2Y > d, for some positive threshold d, then we should sell X and buy Y (since we
%expect X to decrease in price and Y to increase), and similarly, if X - 2Y < -d,
%then we should buy X and sell Y.


Vector error correction model (VECM) introduces this long-run relationship
among a set of cointegrated variables as an error correction term. VECM is a
special case of the vector autorregresive model (VAR) model. VAR model
expresses future values as a linear combination of variables past values.
However, VAR model cannot be used with non-stationary variables. VECM is a
linear model but in terms of variable differences. If cointegration exists,
variable differences are stationary and they introduce an error correction term
which adjusts coefficients to bring the variables back to equilibrium. In
finance, many economic time series turn to be stationary when they are
differentiated and cointegration restrictions often improves
forecasting~\cite{duy1998}. Therefore, VECM has been widely adopted.

Both VECM and VAR model parameters are obtained using ordinary least squares
(OLS) method. OLS has two main problems: is sensitive to errors on input data
and involves many calculations. The former problem is commonly solved using
Ridge Regression (RR) \cite{hoerl1970} which introduces a regularization
parameter that leads to an unbiased estimation with better generalization
capability. The second problem of computational complexity depends on the number
of past values and observations considered.  Recently, online learning
algorithms have been proposed to solve problems with large data sets because of
their simplicity and their ability to update the model when new data is
available. 

In this paper, we propose an online formulation of the VECM called Onlive VECM
(OVECM) based on consideration of only a sliding window of the most recent data.
The algorithm introduces matrix optimizations in order to reduce the number of
operations and also takes into account the fact that cointegration vector space
doesn't experience large changes with small changes in the input data. Moreover,
RR instead of OLS is used to solve VECM. Our method is later tested using four
currency rates from the foreign exchange market with different frequencies.  Model
efectiveness is focused on out-of-sample forecast rather than in-sample fitting.
This criteria allows the OVECM prediction capability to be expressed rather than
just explaining data history. Our method performance is compared with its
optimal offline algorithm.


The next sections are organized as follows: section~\ref{sec:background}
presents the VAR and VECM, the OVECM algorithm proposed is presented in
section~\ref{sec:methodology}. Section~\ref{sec:results} gives a description of
the data used and the tests carried on to show accuracy and time comparison of
our proposal against the traditional VECM and
section~\ref{sec:conclusions} includes conclusions and a proposal for future
tudy.
